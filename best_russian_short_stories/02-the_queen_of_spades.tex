\chapter{\textsc{The Queen Of Spades}\\
\small \hspace{20pt}
By Alexsandr S. Pushkin}

\section{I}


% \lettrine[findent=2pt]{\fbox{\textbf{T}}}{ }
% \lettrine[lraise=0.1, nindent=0em, slope=-.5em]{\sffamily T}{here}
\lettrine[lines=3,lhang=0.11,lraise=0,loversize=0.05]{T}{}%
here was a card party at the rooms of Narumov of the Horse Guards.
The long winter night passed away imperceptibly, and it was five
o'clock in the morning before the company sat down to supper. Those
who had won, ate with a good appetite; the others sat staring absently
at their empty plates. When the champagne appeared, however, the
conversation became more animated, and all took a part in it.

"And how did you fare, Surin?" asked the host.

"Oh, I lost, as usual. I must confess that I am unlucky: I play
mirandole, I always keep cool, I never allow anything to put me out,
and yet I always lose!"

"And you did not once allow yourself to be tempted to back the red?...
Your firmness astonishes me."

"But what do you think of Hermann?" said one of the guests, pointing
to a young Engineer: "he has never had a card in his hand in his life,
he has never in his life laid a wager, and yet he sits here till five
o'clock in the morning watching our play."

"Play interests me very much," said Hermann: "but I am not in the
position to sacrifice the necessary in the hope of winning the
superfluous."

"Hermann is a German: he is economical--that is all!" observed Tomsky.
"But if there is one person that I cannot understand, it is my
grandmother, the Countess Anna Fedotovna."

"How so?" inquired the guests.

"I cannot understand," continued Tomsky, "how it is that my
grandmother does not punt."

"What is there remarkable about an old lady of eighty not punting?"
said Narumov.

"Then you do not know the reason why?"

"No, really; haven't the faintest idea."

"Oh! then listen. About sixty years ago, my grandmother went to Paris,
where she created quite a sensation. People used to run after her to
catch a glimpse of the 'Muscovite Venus.' Richelieu made love to her,
and my grandmother maintains that he almost blew out his brains in
consequence of her cruelty. At that time ladies used to play at faro.
On one occasion at the Court, she lost a very considerable sum to the
Duke of Orleans. On returning home, my grandmother removed the patches
from her face, took off her hoops, informed my grandfather of her loss
at the gaming-table, and ordered him to pay the money. My deceased
grandfather, as far as I remember, was a sort of house-steward to my
grandmother. He dreaded her like fire; but, on hearing of such a heavy
loss, he almost went out of his mind; he calculated the various sums
she had lost, and pointed out to her that in six months she had spent
half a million francs, that neither their Moscow nor Saratov estates
were in Paris, and finally refused point blank to pay the debt. My
grandmother gave him a box on the ear and slept by herself as a sign
of her displeasure. The next day she sent for her husband, hoping that
this domestic punishment had produced an effect upon him, but she
found him inflexible. For the first time in her life, she entered into
reasonings and explanations with him, thinking to be able to convince
him by pointing out to him that there are debts and debts, and that
there is a great difference between a Prince and a coachmaker. But it
was all in vain, my grandfather still remained obdurate. But the
matter did not rest there. My grandmother did not know what to do. She
had shortly before become acquainted with a very remarkable man. You
have heard of Count St. Germain, about whom so many marvellous stories
are told. You know that he represented himself as the Wandering Jew,
as the discoverer of the elixir of life, of the philosopher's stone,
and so forth. Some laughed at him as a charlatan; but Casanova, in his
memoirs, says that he was a spy. But be that as it may, St. Germain,
in spite of the mystery surrounding him, was a very fascinating
person, and was much sought after in the best circles of society. Even
to this day my grandmother retains an affectionate recollection of
him, and becomes quite angry if any one speaks disrespectfully of him.
My grandmother knew that St. Germain had large sums of money at his
disposal. She resolved to have recourse to him, and she wrote a letter
to him asking him to come to her without delay. The queer old man
immediately waited upon her and found her overwhelmed with grief. She
described to him in the blackest colours the barbarity of her husband,
and ended by declaring that her whole hope depended upon his
friendship and amiability.

"St. Germain reflected.

"'I could advance you the sum you want,' said he; 'but I know that you
would not rest easy until you had paid me back, and I should not like
to bring fresh troubles upon you. But there is another way of getting
out of your difficulty: you can win back your money.'

"'But, my dear Count,' replied my grandmother, 'I tell you that I
haven't any money left.'

"'Money is not necessary,' replied St. Germain: 'be pleased to listen
to me.'

"Then he revealed to her a secret, for which each of us would give a
good deal..."

The young officers listened with increased attention. Tomsky lit his
pipe, puffed away for a moment and then continued:

"That same evening my grandmother went to Versailles to the \emph{jeu de la
reine}. The Duke of Orleans kept the bank; my grandmother excused
herself in an off-hand manner for not having yet paid her debt, by
inventing some little story, and then began to play against him. She
chose three cards and played them one after the other: all three won
\emph{sonika}, [Said of a card when it wins or loses in the quickest
possible time.] and my grandmother recovered every farthing that she
had lost."

"Mere chance!" said one of the guests.

"A tale!" observed Hermann.

"Perhaps they were marked cards!" said a third.

"I do not think so," replied Tomsky gravely.

"What!" said Narumov, "you have a grandmother who knows how to hit
upon three lucky cards in succession, and you have never yet succeeded
in getting the secret of it out of her?"

"That's the deuce of it!" replied Tomsky: "she had four sons, one of
whom was my father; all four were determined gamblers, and yet not to
one of them did she ever reveal her secret, although it would not have
been a bad thing either for them or for me. But this is what I heard
from my uncle, Count Ivan Ilyich, and he assured me, on his honour,
that it was true. The late Chaplitzky--the same who died in poverty
after having squandered millions--once lost, in his youth, about three
hundred thousand roubles--to Zorich, if I remember rightly. He was in
despair. My grandmother, who was always very severe upon the
extravagance of young men, took pity, however, upon Chaplitzky. She
gave him three cards, telling him to play them one after the other, at
the same time exacting from him a solemn promise that he would never
play at cards again as long as he lived. Chaplitzky then went to his
victorious opponent, and they began a fresh game. On the first card he
staked fifty thousand rubles and won \emph{sonika}; he doubled the stake
and won again, till at last, by pursuing the same tactics, he won back
more than he had lost ...

"But it is time to go to bed: it is a quarter to six already."

And indeed it was already beginning to dawn: the young men emptied
their glasses and then took leave of each other.



\section{II}


% \lettrine[findent=2pt]{\fbox{\textbf{T}}}{ }
\lettrine[lines=3,lhang=0.11,lraise=0,loversize=0.05]{T}{}%
he old Countess A --- was seated in her dressing-room in front of her
looking-glass. Three waiting maids stood around her. One held a small
pot of rouge, another a box of hair-pins, and the third a tall can
with bright red ribbons. The Countess had no longer the slightest
pretensions to beauty, but she still preserved the habits of her
youth, dressed in strict accordance with the fashion of seventy years
before, and made as long and as careful a toilette as she would have
done sixty years previously. Near the window, at an embroidery frame,
sat a young lady, her ward.

"Good morning, grandmamma," said a young officer, entering the room.
"\emph{Bonjour, Mademoiselle Lise}. Grandmamma, I want to ask you
something."

"What is it, Paul?"

"I want you to let me introduce one of my friends to you, and to allow
me to bring him to the ball on Friday."

"Bring him direct to the ball and introduce him to me there. Were you
at B's yesterday?"

"Yes; everything went off very pleasantly, and dancing was kept up
until five o'clock. How charming Yeletzkaya was!"

"But, my dear, what is there charming about her? Isn't she like her
grandmother, the Princess Daria Petrovna? By the way, she must be very
old, the Princess Daria Petrovna."

"How do you mean, old?" cried Tomsky thoughtlessly; "she died seven
years ago."

The young lady raised her head and made a sign to the young officer.
He then remembered that the old Countess was never to be informed of
the death of any of her contemporaries, and he bit his lips. But the
old Countess heard the news with the greatest indifference.

"Dead!" said she; "and I did not know it. We were appointed maids of
honour at the same time, and when we were presented to the Empress..."

And the Countess for the hundredth time related to her grandson one of
her anecdotes.

"Come, Paul," said she, when she had finished her story, "help me to
get up. Lizanka, where is my snuff-box?"

And the Countess with her three maids went behind a screen to finish
her toilette. Tomsky was left alone with the young lady.

"Who is the gentleman you wish to introduce to the Countess?" asked
Lizaveta Ivanovna in a whisper.

"Narumov. Do you know him?"

"No. Is he a soldier or a civilian?"

"A soldier."

"Is he in the Engineers?"

"No, in the Cavalry. What made you think that he was in the
Engineers?"

The young lady smiled, but made no reply.

"Paul," cried the Countess from behind the screen, "send me some new
novel, only pray don't let it be one of the present day style."

"What do you mean, grandmother?"

"That is, a novel, in which the hero strangles neither his father nor
his mother, and in which there are no drowned bodies. I have a great
horror of drowned persons."

"There are no such novels nowadays. Would you like a Russian one?"

"Are there any Russian novels? Send me one, my dear, pray send me
one!"

"Good-bye, grandmother: I am in a hurry... Good-bye, Lizaveta
Ivanovna. What made you think that Narumov was in the Engineers?"

And Tomsky left the boudoir.

Lizaveta Ivanovna was left alone: she laid aside her work and began to
look out of the window. A few moments afterwards, at a corner house on
the other side of the street, a young officer appeared. A deep blush
covered her cheeks; she took up her work again and bent her head down
over the frame. At the same moment the Countess returned completely
dressed.

"Order the carriage, Lizaveta," said she; "we will go out for a
drive."

Lizaveta arose from the frame and began to arrange her work.

"What is the matter with you, my child, are you deaf?" cried the
Countess. "Order the carriage to be got ready at once."

"I will do so this moment," replied the young lady, hastening into the
ante-room.

A servant entered and gave the Countess some books from Prince Paul
Aleksandrovich.

"Tell him that I am much obliged to him," said the Countess.
"Lizaveta! Lizaveta! Where are you running to?"

"I am going to dress."

"There is plenty of time, my dear. Sit down here. Open the first
volume and read to me aloud."

Her companion took the book and read a few lines.

"Louder," said the Countess. "What is the matter with you, my child?
Have you lost your voice? Wait--give me that footstool--a little
nearer--that will do."

Lizaveta read two more pages. The Countess yawned.

"Put the book down," said she: "what a lot of nonsense! Send it back
to Prince Paul with my thanks... But where is the carriage?"

"The carriage is ready," said Lizaveta, looking out into the street.

"How is it that you are not dressed?" said the Countess: "I must
always wait for you. It is intolerable, my dear!"

Liza hastened to her room. She had not been there two minutes, before
the Countess began to ring with all her might. The three waiting-maids
came running in at one door and the valet at another.

"How is it that you cannot hear me when I ring for you?" said the
Countess. "Tell Lizaveta Ivanovna that I am waiting for her."

Lizaveta returned with her hat and cloak on.

"At last you are here!" said the Countess. "But why such an elaborate
toilette? Whom do you intend to captivate? What sort of weather is it?
It seems rather windy."

"No, your Ladyship, it is very calm," replied the valet.

"You never think of what you are talking about. Open the window. So it
is: windy and bitterly cold. Unharness the horses. Lizaveta, we won't
go out--there was no need for you to deck yourself like that."

"What a life is mine!" thought Lizaveta Ivanovna.

And, in truth, Lizaveta Ivanovna was a very unfortunate creature. "The
bread of the stranger is bitter," says Dante, "and his staircase hard
to climb." But who can know what the bitterness of dependence is so
well as the poor companion of an old lady of quality? The Countess
A had by no means a bad heart, but she was capricious, like a
woman who had been spoilt by the world, as well as being avaricious
and egotistical, like all old people who have seen their best days,
and whose thoughts are with the past and not the present. She
participated in all the vanities of the great world, went to balls,
where she sat in a corner, painted and dressed in old-fashioned style,
like a deformed but indispensable ornament of the ball-room; all the
guests on entering approached her and made a profound bow, as if in
accordance with a set ceremony, but after that nobody took any further
notice of her. She received the whole town at her house, and observed
the strictest etiquette, although she could no longer recognise the
faces of people. Her numerous domestics, growing fat and old in her
ante-chamber and servants' hall, did just as they liked, and vied with
each other in robbing the aged Countess in the most bare-faced manner.
Lizaveta Ivanovna was the martyr of the household. She made tea, and
was reproached with using too much sugar; she read novels aloud to the
Countess, and the faults of the author were visited upon her head; she
accompanied the Countess in her walks, and was held answerable for the
weather or the state of the pavement. A salary was attached to the
post, but she very rarely received it, although she was expected to
dress like everybody else, that is to say, like very few indeed. In
society she played the most pitiable role. Everybody knew her, and
nobody paid her any attention. At balls she danced only when a partner
was wanted, and ladies would only take hold of her arm when it was
necessary to lead her out of the room to attend to their dresses. She
was very self-conscious, and felt her position keenly, and she looked
about her with impatience for a deliverer to come to her rescue; but
the young men, calculating in their giddiness, honoured her with but
very little attention, although Lizaveta Ivanovna was a hundred times
prettier than the bare-faced and cold-hearted marriageable girls
around whom they hovered. Many a time did she quietly slink away from
the glittering but wearisome drawing-room, to go and cry in her own
poor little room, in which stood a screen, a chest of drawers, a
looking-glass and a painted bedstead, and where a tallow candle burnt
feebly in a copper candle-stick.

One morning--this was about two days after the evening party described
at the beginning of this story, and a week previous to the scene at
which we have just assisted--Lizaveta Ivanovna was seated near the
window at her embroidery frame, when, happening to look out into the
street, she caught sight of a young Engineer officer, standing
motionless with his eyes fixed upon her window. She lowered her head
and went on again with her work. About five minutes afterwards she
looked out again--the young officer was still standing in the same
place. Not being in the habit of coquetting with passing officers, she
did not continue to gaze out into the street, but went on sewing for a
couple of hours, without raising her head. Dinner was announced. She
rose up and began to put her embroidery away, but glancing casually
out of the window, she perceived the officer again. This seemed to her
very strange. After dinner she went to the window with a certain
feeling of uneasiness, but the officer was no longer there--and she
thought no more about him.

A couple of days afterwards, just as she was stepping into the
carriage with the Countess, she saw him again. He was standing close
behind the door, with his face half-concealed by his fur collar, but
his dark eyes sparkled beneath his cap. Lizaveta felt alarmed, though
she knew not why, and she trembled as she seated herself in the
carriage.

On returning home, she hastened to the window--the officer was
standing in his accustomed place, with his eyes fixed upon her. She
drew back, a prey to curiosity and agitated by a feeling which was
quite new to her.

From that time forward not a day passed without the young officer
making his appearance under the window at the customary hour, and
between him and her there was established a sort of mute acquaintance.
Sitting in her place at work, she used to feel his approach; and
raising her head, she would look at him longer and longer each day.
The young man seemed to be very grateful to her: she saw with the
sharp eye of youth, how a sudden flush covered his pale cheeks each
time that their glances met. After about a week she commenced to smile
at him...

When Tomsky asked permission of his grandmother the Countess to
present one of his friends to her, the young girl's heart beat
violently. But hearing that Narumov was not an Engineer, she regretted
that by her thoughtless question, she had betrayed her secret to the
volatile Tomsky.

Hermann was the son of a German who had become a naturalised Russian,
and from whom he had inherited a small capital. Being firmly convinced
of the necessity of preserving his independence, Hermann did not touch
his private income, but lived on his pay, without allowing himself the
slightest luxury. Moreover, he was reserved and ambitious, and his
companions rarely had an opportunity of making merry at the expense of
his extreme parsimony. He had strong passions and an ardent
imagination, but his firmness of disposition preserved him from the
ordinary errors of young men. Thus, though a gamester at heart, he
never touched a card, for he considered his position did not allow
him--as he said--"to risk the necessary in the hope of winning the
superfluous," yet he would sit for nights together at the card table
and follow with feverish anxiety the different turns of the game.

The story of the three cards had produced a powerful impression upon
his imagination, and all night long he could think of nothing else.
"If," he thought to himself the following evening, as he walked along
the streets of St. Petersburg, "if the old Countess would but reveal
her secret to me! if she would only tell me the names of the three
winning cards. Why should I not try my fortune? I must get introduced
to her and win her favour--become her lover... But all that will take
time, and she is eighty-seven years old: she might be dead in a week,
in a couple of days even!... But the story itself: can it really be
true?... No! Economy, temperance and industry: those are my three
winning cards; by means of them I shall be able to double my
capital--increase it sevenfold, and procure for myself ease and
independence."

Musing in this manner, he walked on until he found himself in one of
the principal streets of St. Petersburg, in front of a house of
antiquated architecture. The street was blocked with equipages;
carriages one after the other drew up in front of the brilliantly
illuminated doorway. At one moment there stepped out on to the
pavement the well-shaped little foot of some young beauty, at another
the heavy boot of a cavalry officer, and then the silk stockings and
shoes of a member of the diplomatic world. Furs and cloaks passed in
rapid succession before the gigantic porter at the entrance.

Hermann stopped. "Whose house is this?" he asked of the watchman at
the corner.

"The Countess A's," replied the watchman.

Hermann started. The strange story of the three cards again presented
itself to his imagination. He began walking up and down before the
house, thinking of its owner and her strange secret. Returning late to
his modest lodging, he could not go to sleep for a long time, and when
at last he did doze off, he could dream of nothing but cards, green
tables, piles of banknotes and heaps of ducats. He played one card
after the other, winning uninterruptedly, and then he gathered up the
gold and filled his pockets with the notes. When he woke up late the
next morning, he sighed over the loss of his imaginary wealth, and
then sallying out into the town, he found himself once more in front
of the Countess's residence. Some unknown power seemed to have
attracted him thither. He stopped and looked up at the windows. At one
of these he saw a head with luxuriant black hair, which was bent down
probably over some book or an embroidery frame. The head was raised.
Hermann saw a fresh complexion and a pair of dark eyes. That moment
decided his fate.



\section{III}

% \lettrine[findent=2pt]{\fbox{\textbf{L}}}{ }
\lettrine[lines=3,lhang=0.11,lraise=0,loversize=0.05]{L}{}%
izaveta Ivanovna had scarcely taken off her hat and cloak, when the
Countess sent for her and again ordered her to get the carriage ready.
The vehicle drew up before the door, and they prepared to take their
seats. Just at the moment when two footmen were assisting the old lady
to enter the carriage, Lizaveta saw her Engineer standing close beside
the wheel; he grasped her hand; alarm caused her to lose her presence
of mind, and the young man disappeared--but not before he had left a
letter between her fingers. She concealed it in her glove, and during
the whole of the drive she neither saw nor heard anything. It was the
custom of the Countess, when out for an airing in her carriage, to be
constantly asking such questions as: "Who was that person that met us
just now? What is the name of this bridge? What is written on that
signboard?" On this occasion, however, Lizaveta returned such vague
and absurd answers, that the Countess became angry with her.

"What is the matter with you, my dear?" she exclaimed. "Have you taken
leave of your senses, or what is it? Do you not hear me or understand
what I say?... Heaven be thanked, I am still in my right mind and
speak plainly enough!"

Lizaveta Ivanovna did not hear her. On returning home she ran to her
room, and drew the letter out of her glove: it was not sealed.
Lizaveta read it. The letter contained a declaration of love; it was
tender, respectful, and copied word for word from a German novel. But
Lizaveta did not know anything of the German language, and she was
quite delighted.

For all that, the letter caused her to feel exceedingly uneasy. For
the first time in her life she was entering into secret and
confidential relations with a young man. His boldness alarmed her. She
reproached herself for her imprudent behaviour, and knew not what to
do. Should she cease to sit at the window and, by assuming an
appearance of indifference towards him, put a check upon the young
officer's desire for further acquaintance with her? Should she send
his letter back to him, or should she answer him in a cold and decided
manner? There was nobody to whom she could turn in her perplexity, for
she had neither female friend nor adviser... At length she resolved to
reply to him.

She sat down at her little writing-table, took pen and paper, and
began to think. Several times she began her letter, and then tore it
up: the way she had expressed herself seemed to her either too
inviting or too cold and decisive. At last she succeeded in writing a
few lines with which she felt satisfied.

"I am convinced," she wrote, "that your intentions are honourable, and
that you do not wish to offend me by any imprudent behaviour, but our
acquaintance must not begin in such a manner. I return you your
letter, and I hope that I shall never have any cause to complain of
this undeserved slight."

The next day, as soon as Hermann made his appearance, Lizaveta rose
from her embroidery, went into the drawing-room, opened the ventilator
and threw the letter into the street, trusting that the young officer
would have the perception to pick it up.

Hermann hastened forward, picked it up and then repaired to a
confectioner's shop. Breaking the seal of the envelope, he found
inside it his own letter and Lizaveta's reply. He had expected this,
and he returned home, his mind deeply occupied with his intrigue.

Three days afterwards, a bright-eyed young girl from a milliner's
establishment brought Lizaveta a letter. Lizaveta opened it with great
uneasiness, fearing that it was a demand for money, when suddenly she
recognised Hermann's hand-writing.

"You have made a mistake, my dear," said she: "this letter is not for
me."

"Oh, yes, it is for you," replied the girl, smiling very knowingly.
"Have the goodness to read it."

Lizaveta glanced at the letter. Hermann requested an interview.

"It cannot be," she cried, alarmed at the audacious request, and the
manner in which it was made. "This letter is certainly not for me."

And she tore it into fragments.

"If the letter was not for you, why have you torn it up?" said the
girl. "I should have given it back to the person who sent it."

"Be good enough, my dear," said Lizaveta, disconcerted by this remark,
"not to bring me any more letters for the future, and tell the person
who sent you that he ought to be ashamed..."

But Hermann was not the man to be thus put off. Every day Lizaveta
received from him a letter, sent now in this way, now in that. They
were no longer translated from the German. Hermann wrote them under
the inspiration of passion, and spoke in his own language, and they
bore full testimony to the inflexibility of his desire and the
disordered condition of his uncontrollable imagination. Lizaveta no
longer thought of sending them back to him: she became intoxicated
with them and began to reply to them, and little by little her answers
became longer and more affectionate. At last she threw out of the
window to him the following letter:

"This evening there is going to be a ball at the Embassy. The Countess
will be there. We shall remain until two o'clock. You have now an
opportunity of seeing me alone. As soon as the Countess is gone, the
servants will very probably go out, and there will be nobody left but
the Swiss, but he usually goes to sleep in his lodge. Come about
half-past eleven. Walk straight upstairs. If you meet anybody in the
ante-room, ask if the Countess is at home. You will be told 'No,' in
which case there will be nothing left for you to do but to go away
again. But it is most probable that you will meet nobody. The
maidservants will all be together in one room. On leaving the
ante-room, turn to the left, and walk straight on until you reach the
Countess's bedroom. In the bedroom, behind a screen, you will find two
doors: the one on the right leads to a cabinet, which the Countess
never enters; the one on the left leads to a corridor, at the end of
which is a little winding staircase; this leads to my room."

Hermann trembled like a tiger, as he waited for the appointed time to
arrive. At ten o'clock in the evening he was already in front of the
Countess's house. The weather was terrible; the wind blew with great
violence; the sleety snow fell in large flakes; the lamps emitted a
feeble light, the streets were deserted; from time to time a sledge,
drawn by a sorry-looking hack, passed by, on the look-out for a
belated passenger. Hermann was enveloped in a thick overcoat, and felt
neither wind nor snow.

At last the Countess's carriage drew up. Hermann saw two footmen carry
out in their arms the bent form of the old lady, wrapped in sable fur,
and immediately behind her, clad in a warm mantle, and with her head
ornamented with a wreath of fresh flowers, followed Lizaveta. The door
was closed. The carriage rolled away heavily through the yielding
snow. The porter shut the street-door; the windows became dark.

Hermann began walking up and down near the deserted house; at length
he stopped under a lamp, and glanced at his watch: it was twenty
minutes past eleven. He remained standing under the lamp, his eyes
fixed upon the watch, impatiently waiting for the remaining minutes to
pass. At half-past eleven precisely, Hermann ascended the steps of the
house, and made his way into the brightly-illuminated vestibule. The
porter was not there. Hermann hastily ascended the staircase, opened
the door of the ante-room and saw a footman sitting asleep in an
antique chair by the side of a lamp. With a light firm step Hermann
passed by him. The drawing-room and dining-room were in darkness, but
a feeble reflection penetrated thither from the lamp in the ante-room.

Hermann reached the Countess's bedroom. Before a shrine, which was
full of old images, a golden lamp was burning. Faded stuffed chairs
and divans with soft cushions stood in melancholy symmetry around the
room, the walls of which were hung with China silk. On one side of the
room hung two portraits painted in Paris by Madame Lebrun. One of
these represented a stout, red-faced man of about forty years of age
in a bright-green uniform and with a star upon his breast; the
other--a beautiful young woman, with an aquiline nose, forehead curls
and a rose in her powdered hair. In the corners stood porcelain
shepherds and shepherdesses, dining-room clocks from the workshop of
the celebrated Lefroy, bandboxes, roulettes, fans and the various
playthings for the amusement of ladies that were in vogue at the end
of the last century, when Montgolfier's balloons and Mesmer's
magnetism were the rage. Hermann stepped behind the screen. At the
back of it stood a little iron bedstead; on the right was the door
which led to the cabinet; on the left--the other which led to the
corridor. He opened the latter, and saw the little winding staircase
which led to the room of the poor companion... But he retraced his
steps and entered the dark cabinet.

The time passed slowly. All was still. The clock in the drawing-room
struck twelve; the strokes echoed through the room one after the
other, and everything was quiet again. Hermann stood leaning against
the cold stove. He was calm; his heart beat regularly, like that of a
man resolved upon a dangerous but inevitable undertaking. One o'clock
in the morning struck; then two; and he heard the distant noise of
carriage-wheels. An involuntary agitation took possession of him. The
carriage drew near and stopped. He heard the sound of the
carriage-steps being let down. All was bustle within the house. The
servants were running hither and thither, there was a confusion of
voices, and the rooms were lit up. Three antiquated chamber-maids
entered the bedroom, and they were shortly afterwards followed by the
Countess who, more dead than alive, sank into a Voltaire armchair.
Hermann peeped through a chink. Lizaveta Ivanovna passed close by him,
and he heard her hurried steps as she hastened up the little spiral
staircase. For a moment his heart was assailed by something like a
pricking of conscience, but the emotion was only transitory, and his
heart became petrified as before.

The Countess began to undress before her looking-glass. Her
rose-bedecked cap was taken off, and then her powdered wig was removed
from off her white and closely-cut hair. Hairpins fell in showers
around her. Her yellow satin dress, brocaded with silver, fell down at
her swollen feet.

Hermann was a witness of the repugnant mysteries of her toilette; at
last the Countess was in her night-cap and dressing-gown, and in this
costume, more suitable to her age, she appeared less hideous and
deformed.

Like all old people in general, the Countess suffered from
sleeplessness. Having undressed, she seated herself at the window in a
Voltaire armchair and dismissed her maids. The candles were taken
away, and once more the room was left with only one lamp burning in
it. The Countess sat there looking quite yellow, mumbling with her
flaccid lips and swaying to and fro. Her dull eyes expressed complete
vacancy of mind, and, looking at her, one would have thought that the
rocking of her body was not a voluntary action of her own, but was
produced by the action of some concealed galvanic mechanism.

Suddenly the death-like face assumed an inexplicable expression. The
lips ceased to tremble, the eyes became animated: before the Countess
stood an unknown man.

"Do not be alarmed, for Heaven's sake, do not be alarmed!" said he in
a low but distinct voice. "I have no intention of doing you any harm,
I have only come to ask a favour of you."

The old woman looked at him in silence, as if she had not heard what
he had said. Hermann thought that she was deaf, and bending down
towards her ear, he repeated what he had said. The aged Countess
remained silent as before.

"You can insure the happiness of my life," continued Hermann, "and it
will cost you nothing. I know that you can name three cards in
order--"

Hermann stopped. The Countess appeared now to understand what he
wanted; she seemed as if seeking for words to reply.

"It was a joke," she replied at last: "I assure you it was only a
joke."

"There is no joking about the matter," replied Hermann angrily.
"Remember Chaplitzky, whom you helped to win."

The Countess became visibly uneasy. Her features expressed strong
emotion, but they quickly resumed their former immobility.

"Can you not name me these three winning cards?" continued Hermann.

The Countess remained silent; Hermann continued:

"For whom are you preserving your secret? For your grandsons? They are
rich enough without it; they do not know the worth of money. Your
cards would be of no use to a spendthrift. He who cannot preserve his
paternal inheritance, will die in want, even though he had a demon at
his service. I am not a man of that sort; I know the value of money.
Your three cards will not be thrown away upon me. Come!"...

He paused and tremblingly awaited her reply. The Countess remained
silent; Hermann fell upon his knees.

"If your heart has ever known the feeling of love," said he, "if you
remember its rapture, if you have ever smiled at the cry of your
new-born child, if any human feeling has ever entered into your
breast, I entreat you by the feelings of a wife, a lover, a mother, by
all that is most sacred in life, not to reject my prayer. Reveal to me
your secret. Of what use is it to you?... May be it is connected with
some terrible sin with the loss of eternal salvation, with some
bargain with the devil... Reflect,--you are old; you have not long to
live--I am ready to take your sins upon my soul. Only reveal to me
your secret. Remember that the happiness of a man is in your hands,
that not only I, but my children, and grandchildren will bless your
memory and reverence you as a saint..."

The old Countess answered not a word.

Hermann rose to his feet.

"You old hag!" he exclaimed, grinding his teeth, "then I will make you
answer!"

With these words he drew a pistol from his pocket.

At the sight of the pistol, the Countess for the second time exhibited
strong emotion. She shook her head and raised her hands as if to
protect herself from the shot... then she fell backwards and remained
motionless.

"Come, an end to this childish nonsense!" said Hermann, taking hold of
her hand. "I ask you for the last time: will you tell me the names of
your three cards, or will you not?"

The Countess made no reply. Hermann perceived that she was dead!



\section{IV}


% \lettrine[findent=2pt]{\fbox{\textbf{L}}}{ }
\lettrine[lines=3,lhang=0.11,lraise=0,loversize=0.05]{L}{}%
izaveta Ivanovna was sitting in her room, still in her ball dress,
lost in deep thought. On returning home, she had hastily dismissed the
chambermaid who very reluctantly came forward to assist her, saying
that she would undress herself, and with a trembling heart had gone up
to her own room, expecting to find Hermann there, but yet hoping not
to find him. At the first glance she convinced herself that he was not
there, and she thanked her fate for having prevented him keeping the
appointment. She sat down without undressing, and began to recall to
mind all the circumstances which in so short a time had carried her so
far. It was not three weeks since the time when she first saw the
young officer from the window--and yet she was already in
correspondence with him, and he had succeeded in inducing her to grant
him a nocturnal interview! She knew his name only through his having
written it at the bottom of some of his letters; she had never spoken
to him, had never heard his voice, and had never heard him spoken of
until that evening. But, strange to say, that very evening at the
ball, Tomsky, being piqued with the young Princess Pauline N, who,
contrary to her usual custom, did not flirt with him, wished to
revenge himself by assuming an air of indifference: he therefore
engaged Lizaveta Ivanovna and danced an endless mazurka with her.
During the whole of the time he kept teasing her about her partiality
for Engineer officers; he assured her that he knew far more than she
imagined, and some of his jests were so happily aimed, that Lizaveta
thought several times that her secret was known to him.

"From whom have you learnt all this?" she asked, smiling.

"From a friend of a person very well known to you," replied Tomsky,
"from a very distinguished man."

"And who is this distinguished man?"

"His name is Hermann."

Lizaveta made no reply; but her hands and feet lost all sense of
feeling.

"This Hermann," continued Tomsky, "is a man of romantic personality.
He has the profile of a Napoleon, and the soul of a Mephistopheles. I
believe that he has at least three crimes upon his conscience... How
pale you have become!"

"I have a headache... But what did this Hermann--or whatever his name
is--tell you?"

"Hermann is very much dissatisfied with his friend: he says that in
his place he would act very differently... I even think that Hermann
himself has designs upon you; at least, he listens very attentively to
all that his friend has to say about you."

"And where has he seen me?"

"In church, perhaps; or on the parade--God alone knows where. It may
have been in your room, while you were asleep, for there is nothing
that he--"

Three ladies approaching him with the question: "\emph{oubli ou regret}?"
interrupted the conversation, which had become so tantalisingly
interesting to Lizaveta.

The lady chosen by Tomsky was the Princess Pauline herself. She
succeeded in effecting a reconciliation with him during the numerous
turns of the dance, after which he conducted her to her chair. On
returning to his place, Tomsky thought no more either of Hermann or
Lizaveta. She longed to renew the interrupted conversation, but the
mazurka came to an end, and shortly afterwards the old Countess took
her departure.

Tomsky's words were nothing more than the customary small talk of the
dance, but they sank deep into the soul of the young dreamer. The
portrait, sketched by Tomsky, coincided with the picture she had
formed within her own mind, and thanks to the latest romances, the
ordinary countenance of her admirer became invested with attributes
capable of alarming her and fascinating her imagination at the same
time. She was now sitting with her bare arms crossed and with her
head, still adorned with flowers, sunk upon her uncovered bosom.
Suddenly the door opened and Hermann entered. She shuddered.

"Where were you?" she asked in a terrified whisper.

"In the old Countess's bedroom," replied Hermann: "I have just left
her. The Countess is dead."

"My God! What do you say?"

"And I am afraid," added Hermann, "that I am the cause of her death."

Lizaveta looked at him, and Tomsky's words found an echo in her soul:
"This man has at least three crimes upon his conscience!" Hermann sat
down by the window near her, and related all that had happened.

Lizaveta listened to him in terror. So all those passionate letters,
those ardent desires, this bold obstinate pursuit--all this was not
love! Money--that was what his soul yearned for! She could not satisfy
his desire and make him happy! The poor girl had been nothing but
the blind tool of a robber, of the murderer of her aged
benefactress!... She wept bitter tears of agonised repentance. Hermann
gazed at her in silence: his heart, too, was a prey to violent
emotion, but neither the tears of the poor girl, nor the wonderful
charm of her beauty, enhanced by her grief, could produce any
impression upon his hardened soul. He felt no pricking of conscience
at the thought of the dead old woman. One thing only grieved him: the
irreparable loss of the secret from which he had expected to obtain
great wealth.

"You are a monster!" said Lizaveta at last.

"I did not wish for her death," replied Hermann: "my pistol was not
loaded."

Both remained silent.

The day began to dawn. Lizaveta extinguished her candle: a pale light
illumined her room. She wiped her tear-stained eyes and raised them
towards Hermann: he was sitting near the window, with his arms crossed
and with a fierce frown upon his forehead. In this attitude he bore a
striking resemblance to the portrait of Napoleon. This resemblance
struck Lizaveta even.

"How shall I get you out of the house?" said she at last. "I thought
of conducting you down the secret staircase, but in that case it would
be necessary to go through the Countess's bedroom, and I am afraid."

"Tell me how to find this secret staircase--I will go alone."

Lizaveta arose, took from her drawer a key, handed it to Hermann and
gave him the necessary instructions. Hermann pressed her cold, limp
hand, kissed her bowed head, and left the room.

He descended the winding staircase, and once more entered the
Countess's bedroom. The dead old lady sat as if petrified; her face
expressed profound tranquillity. Hermann stopped before her, and gazed
long and earnestly at her, as if he wished to convince himself of the
terrible reality; at last he entered the cabinet, felt behind the
tapestry for the door, and then began to descend the dark staircase,
filled with strange emotions. "Down this very staircase," thought he,
"perhaps coming from the very same room, and at this very same hour
sixty years ago, there may have glided, in an embroidered coat, with
his hair dressed \emph{à l'oiseau royal} and pressing to his heart his
three-cornered hat, some young gallant, who has long been mouldering
in the grave, but the heart of his aged mistress has only to-day
ceased to beat..."

At the bottom of the staircase Hermann found a door, which he opened
with a key, and then traversed a corridor which conducted him into the
street.



\section{V}


% \lettrine[findent=2pt]{\fbox{\textbf{T}}}{ }
\lettrine[lines=3,lhang=0.11,lraise=0,loversize=0.05]{T}{}%
hree days after the fatal night, at nine o'clock in the morning,
Hermann repaired to the Convent of , where the last honours were
to be paid to the mortal remains of the old Countess. Although feeling
no remorse, he could not altogether stifle the voice of conscience,
which said to him: "You are the murderer of the old woman!" In spite
of his entertaining very little religious belief, he was exceedingly
superstitious; and believing that the dead Countess might exercise an
evil influence on his life, he resolved to be present at her obsequies
in order to implore her pardon.

The church was full. It was with difficulty that Hermann made his way
through the crowd of people. The coffin was placed upon a rich
catafalque beneath a velvet baldachin. The deceased Countess lay
within it, with her hands crossed upon her breast, with a lace cap
upon her head and dressed in a white satin robe. Around the catafalque
stood the members of her household: the servants in black \emph{caftans},
with armorial ribbons upon their shoulders, and candles in their
hands; the relatives--children, grandchildren, and
great-grandchildren--in deep mourning.

Nobody wept; tears would have been \emph{une affectation}. The Countess was
so old, that her death could have surprised nobody, and her relatives
had long looked upon her as being out of the world. A famous preacher
pronounced the funeral sermon. In simple and touching words he
described the peaceful passing away of the righteous, who had passed
long years in calm preparation for a Christian end. "The angel of
death found her," said the orator, "engaged in pious meditation and
waiting for the midnight bridegroom."

The service concluded amidst profound silence. The relatives went
forward first to take farewell of the corpse. Then followed the
numerous guests, who had come to render the last homage to her who for
so many years had been a participator in their frivolous amusements.
After these followed the members of the Countess's household. The last
of these was an old woman of the same age as the deceased. Two young
women led her forward by the hand. She had not strength enough to bow
down to the ground--she merely shed a few tears and kissed the cold
hand of her mistress.

Hermann now resolved to approach the coffin. He knelt down upon the
cold stones and remained in that position for some minutes; at last he
arose, as pale as the deceased Countess herself; he ascended the steps
of the catafalque and bent over the corpse... At that moment it seemed
to him that the dead woman darted a mocking look at him and winked
with one eye. Hermann started back, took a false step and fell to the
ground. Several persons hurried forward and raised him up. At the same
moment Lizaveta Ivanovna was borne fainting into the porch of the
church. This episode disturbed for some minutes the solemnity of the
gloomy ceremony. Among the congregation arose a deep murmur, and a
tall thin chamberlain, a near relative of the deceased, whispered in
the ear of an Englishman who was standing near him, that the young
officer was a natural son of the Countess, to which the Englishman
coldly replied: "Oh!"

During the whole of that day, Hermann was strangely excited. Repairing
to an out-of-the-way restaurant to dine, he drank a great deal of
wine, contrary to his usual custom, in the hope of deadening his
inward agitation. But the wine only served to excite his imagination
still more. On returning home, he threw himself upon his bed without
undressing, and fell into a deep sleep.

When he woke up it was already night, and the moon was shining into
the room. He looked at his watch: it was a quarter to three. Sleep had
left him; he sat down upon his bed and thought of the funeral of the
old Countess.

At that moment somebody in the street looked in at his window, and
immediately passed on again. Hermann paid no attention to this
incident. A few moments afterwards he heard the door of his ante-room
open. Hermann thought that it was his orderly, drunk as usual,
returning from some nocturnal expedition, but presently he heard
footsteps that were unknown to him: somebody was walking softly over
the floor in slippers. The door opened, and a woman dressed in white,
entered the room. Hermann mistook her for his old nurse, and wondered
what could bring her there at that hour of the night. But the white
woman glided rapidly across the room and stood before him--and Hermann
recognised the Countess!

"I have come to you against my wish," she said in a firm voice: "but I
have been ordered to grant your request. Three, seven, ace, will win
for you if played in succession, but only on these conditions: that
you do not play more than one card in twenty-four hours, and that you
never play again during the rest of your life. I forgive you my death,
on condition that you marry my companion, Lizaveta Ivanovna."

With these words she turned round very quietly, walked with a
shuffling gait towards the door and disappeared. Hermann heard the
street-door open and shut, and again he saw some one look in at him
through the window.

For a long time Hermann could not recover himself. He then rose up and
entered the next room. His orderly was lying asleep upon the floor,
and he had much difficulty in waking him. The orderly was drunk as
usual, and no information could be obtained from him. The street-door
was locked. Hermann returned to his room, lit his candle, and wrote
down all the details of his vision.



\section{VI}


% \lettrine[findent=2pt]{\fbox{\textbf{T}}}{ }
\lettrine[lines=3,lhang=0.11,lraise=0,loversize=0.05]{T}{wo}
fixed ideas can no more exist together in the moral world than two
bodies can occupy one and the same place in the physical world.
"Three, seven, ace," soon drove out of Hermann's mind the thought of
the dead Countess. "Three, seven, ace," were perpetually running
through his head and continually being repeated by his lips. If he saw
a young girl, he would say: "How slender she is! quite like the three
of hearts." If anybody asked: "What is the time?" he would reply:
"Five minutes to seven." Every stout man that he saw reminded him of
the ace. "Three, seven, ace" haunted him in his sleep, and assumed all
possible shapes. The threes bloomed before him in the forms of
magnificent flowers, the sevens were represented by Gothic portals,
and the aces became transformed into gigantic spiders. One thought
alone occupied his whole mind--to make a profitable use of the secret
which he had purchased so dearly. He thought of applying for a
furlough so as to travel abroad. He wanted to go to Paris and tempt
fortune in some of the public gambling-houses that abounded there.
Chance spared him all this trouble.

There was in Moscow a society of rich gamesters, presided over by the
celebrated Chekalinsky, who had passed all his life at the card-table
and had amassed millions, accepting bills of exchange for his winnings
and paying his losses in ready money. His long experience secured for
him the confidence of his companions, and his open house, his famous
cook, and his agreeable and fascinating manners gained for him the
respect of the public. He came to St. Petersburg. The young men of the
capital flocked to his rooms, forgetting balls for cards, and
preferring the emotions of faro to the seductions of flirting. Narumov
conducted Hermann to Chekalinsky's residence.

They passed through a suite of magnificent rooms, filled with
attentive domestics. The place was crowded. Generals and Privy
Counsellors were playing at whist; young men were lolling carelessly
upon the velvet-covered sofas, eating ices and smoking pipes. In the
drawing-room, at the head of a long table, around which were assembled
about a score of players, sat the master of the house keeping the
bank. He was a man of about sixty years of age, of a very dignified
appearance; his head was covered with silvery-white hair; his full,
florid countenance expressed good-nature, and his eyes twinkled with a
perpetual smile. Narumov introduced Hermann to him. Chekalinsky shook
him by the hand in a friendly manner, requested him not to stand on
ceremony, and then went on dealing.

The game occupied some time. On the table lay more than thirty cards.
Chekalinsky paused after each throw, in order to give the players time
to arrange their cards and note down their losses, listened politely
to their requests, and more politely still, put straight the corners
of cards that some player's hand had chanced to bend. At last the game
was finished. Chekalinsky shuffled the cards and prepared to deal
again.

"Will you allow me to take a card?" said Hermann, stretching out his
hand from behind a stout gentleman who was punting.

Chekalinsky smiled and bowed silently, as a sign of acquiescence.
Narumov laughingly congratulated Hermann on his abjuration of that
abstention from cards which he had practised for so long a period, and
wished him a lucky beginning.

"Stake!" said Hermann, writing some figures with chalk on the back of
his card.

"How much?" asked the banker, contracting the muscles of his eyes;
"excuse me, I cannot see quite clearly."

"Forty-seven thousand rubles," replied Hermann.

At these words every head in the room turned suddenly round, and all
eyes were fixed upon Hermann.

"He has taken leave of his senses!" thought Narumov.

"Allow me to inform you," said Chekalinsky, with his eternal smile,
"that you are playing very high; nobody here has ever staked more than
two hundred and seventy-five rubles at once."

"Very well," replied Hermann; "but do you accept my card or not?"

Chekalinsky bowed in token of consent.

"I only wish to observe," said he, "that although I have the greatest
confidence in my friends, I can only play against ready money. For my
own part, I am quite convinced that your word is sufficient, but for
the sake of the order of the game, and to facilitate the reckoning up,
I must ask you to put the money on your card."

Hermann drew from his pocket a bank-note and handed it to Chekalinsky,
who, after examining it in a cursory manner, placed it on Hermann's
card.

He began to deal. On the right a nine turned up, and on the left a
three.

"I have won!" said Hermann, showing his card.

A murmur of astonishment arose among the players. Chekalinsky frowned,
but the smile quickly returned to his face.

"Do you wish me to settle with you?" he said to Hermann.

"If you please," replied the latter.

Chekalinsky drew from his pocket a number of banknotes and paid at
once. Hermann took up his money and left the table. Narumov could not
recover from his astonishment. Hermann drank a glass of lemonade and
returned home.

The next evening he again repaired to Chekalinsky's. The host was
dealing. Hermann walked up to the table; the punters immediately made
room for him. Chekalinsky greeted him with a gracious bow.

Hermann waited for the next deal, took a card and placed upon it his
forty-seven thousand roubles, together with his winnings of the
previous evening.

Chekalinsky began to deal. A knave turned up on the right, a seven on
the left.

Hermann showed his seven.

There was a general exclamation. Chekalinsky was evidently ill at
ease, but he counted out the ninety-four thousand rubles and handed
them over to Hermann, who pocketed them in the coolest manner possible
and immediately left the house.

The next evening Hermann appeared again at the table. Every one was
expecting him. The generals and Privy Counsellors left their whist in
order to watch such extraordinary play. The young officers quitted
their sofas, and even the servants crowded into the room. All pressed
round Hermann. The other players left off punting, impatient to see
how it would end. Hermann stood at the table and prepared to play
alone against the pale, but still smiling Chekalinsky. Each opened a
pack of cards. Chekalinsky shuffled. Hermann took a card and covered
it with a pile of bank-notes. It was like a duel. Deep silence reigned
around.

Chekalinsky began to deal; his hands trembled. On the right a queen
turned up, and on the left an ace.

"Ace has won!" cried Hermann, showing his card.

"Your queen has lost," said Chekalinsky, politely.

Hermann started; instead of an ace, there lay before him the queen of
spades! He could not believe his eyes, nor could he understand how he
had made such a mistake.

At that moment it seemed to him that the queen of spades smiled
ironically and winked her eye at him. He was struck by her remarkable
resemblance...

"The old Countess!" he exclaimed, seized with terror.

Chekalinsky gathered up his winnings. For some time, Hermann remained
perfectly motionless. When at last he left the table, there was a
general commotion in the room.

"Splendidly punted!" said the players. Chekalinsky shuffled the cards
afresh, and the game went on as usual.

\begin{center}
    \begin{verbatim}
       *       *       *       *       *
   \end{verbatim}
\end{center}

Hermann went out of his mind, and is now confined in room Number 17 of
the Obukhov Hospital. He never answers any questions, but he
constantly mutters with unusual rapidity: "Three, seven, ace!" "Three,
seven, queen!"

Lizaveta Ivanovna has married a very amiable young man, a son of the
former steward of the old Countess. He is in the service of the State
somewhere, and is in receipt of a good income. Lizaveta is also
supporting a poor relative.

Tomsky has been promoted to the rank of captain, and has become the
husband of the Princess Pauline.
