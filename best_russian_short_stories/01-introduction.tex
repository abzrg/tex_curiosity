\chapter{\textsc{Introduction}}


% \lettrine[findent=2pt]{\fbox{\textbf{C}}}{ }
\lettrine[lines=3,lhang=0.11,lraise=0,loversize=0.05]{C}{}%
onceive the joy of a lover of nature who, leaving the art galleries,
wanders out among the trees and wild flowers and birds that the
pictures of the galleries have sentimentalised. It is some such joy
that the man who truly loves the noblest in letters feels when tasting
for the first time the simple delights of Russian literature. French
and English and German authors, too, occasionally, offer works of
lofty, simple naturalness; but the very keynote to the whole of
Russian literature is simplicity, naturalness, veraciousness.

Another essentially Russian trait is the quite unaffected conception
that the lowly are on a plane of equality with the so-called upper
classes. When the Englishman Dickens wrote with his profound pity and
understanding of the poor, there was yet a bit; of remoteness,
perhaps, even, a bit of caricature, in his treatment of them. He
showed their sufferings to the rest of the world with a ``Behold how
the other half lives!" The Russian writes of the poor, as it were,
from within, as one of them, with no eye to theatrical effect upon the
well-to-do. There is no insistence upon peculiar virtues or vices. The
poor are portrayed just as they are, as human beings like the rest of
us. A democratic spirit is reflected, breathing a broad humanity, a
true universality, an unstudied generosity that proceed not from the
intellectual conviction that to understand all is to forgive all, but
from an instinctive feeling that no man has the right to set himself
up as a judge over another, that one can only observe and record.

In 1834 two short stories appeared, \emph{The Queen of Spades}, by Pushkin,
and \emph{The Cloak}, by Gogol. The first was a finishing-off of the old,
outgoing style of romanticism, the other was the beginning of the new,
the characteristically Russian style. We read Pushkin's \emph{Queen of
Spades}, the first story in the volume, and the likelihood is we shall
enjoy it greatly. ``But why is it Russian?" we ask. The answer is, ``It
is not Russian." It might have been printed in an American magazine
over the name of John Brown. But, now, take the very next story in the
volume, \emph{The Cloak}. ``Ah," you exclaim, ``a genuine Russian story,
Surely. You cannot palm it off on me over the name of Jones or Smith."
Why? Because \emph{The Cloak} for the first time strikes that truly Russian
note of deep sympathy with the disinherited. It is not yet wholly free
from artificiality, and so is not yet typical of the purely realistic
fiction that reached its perfected development in Turgenev and
Tolstoy.

Though Pushkin heads the list of those writers who made the literature
of their country world-famous, he was still a romanticist, in the
universal literary fashion of his day. However, he already gave strong
indication of the peculiarly Russian genius for naturalness or
realism, and was a true Russian in his simplicity of style. In no
sense an innovator, but taking the cue for his poetry from Byron and
for his prose from the romanticism current at that period, he was not
in advance of his age. He had a revolutionary streak in his nature, as
his \emph{Ode to Liberty} and other bits of verse and his intimacy with the
Decembrist rebels show. But his youthful fire soon died down, and he
found it possible to accommodate himself to the life of a Russian high
functionary and courtier under the severe despot Nicholas I, though,
to be sure, he always hated that life. For all his flirting with
revolutionarism, he never displayed great originality or depth of
thought. He was simply an extraordinarily gifted author, a perfect
versifier, a wondrous lyrist, and a delicious raconteur, endowed with
a grace, ease and power of expression that delighted even the exacting
artistic sense of Turgenev. To him aptly applies the dictum of
Socrates: ``Not by wisdom do the poets write poetry, but by a sort of
genius and inspiration." I do not mean to convey that as a thinker
Pushkin is to be despised. Nevertheless, it is true that he would
occupy a lower position in literature did his reputation depend upon
his contributions to thought and not upon his value as an artist.

``We are all descended from Gogol's \emph{Cloak}," said a Russian writer.
And Dostoyevsky's novel, \emph{Poor People}, which appeared ten years
later, is, in a way, merely an extension of Gogol's shorter tale. In
Dostoyevsky, indeed, the passion for the common people and the
all-embracing, all-penetrating pity for suffering humanity reach their
climax. He was a profound psychologist and delved deeply into the
human soul, especially in its abnormal and diseased aspects. Between
scenes of heart-rending, abject poverty, injustice, and wrong, and the
torments of mental pathology, he managed almost to exhaust the whole
range of human woe. And he analysed this misery with an intensity of
feeling and a painstaking regard for the most harrowing details that
are quite upsetting to normally constituted nerves. Yet all the
horrors must be forgiven him because of the motive inspiring them--an
overpowering love and the desire to induce an equal love in others. It
is not horror for horror's sake, not a literary \emph{tour de force}, as in
Poe, but horror for a high purpose, for purification through
suffering, which was one of the articles of Dostoyevsky's faith.

Following as a corollary from the love and pity for mankind that make
a leading element in Russian literature, is a passionate search for
the means of improving the lot of humanity, a fervent attachment to
social ideas and ideals. A Russian author is more ardently devoted to
a cause than an American short-story writer to a plot. This, in turn,
is but a reflection of the spirit of the Russian people, especially of
the intellectuals. The Russians take literature perhaps more seriously
than any other nation. To them books are not a mere diversion. They
demand that fiction and poetry be a true mirror of life and be of
service to life. A Russian author, to achieve the highest recognition,
must be a thinker also. He need not necessarily be a finished artist.
Everything is subordinated to two main requirements--humanitarian
ideals and fidelity to life. This is the secret of the marvellous
simplicity of Russian-literary art. Before the supreme function of
literature, the Russian writer stands awed and humbled. He knows he
cannot cover up poverty of thought, poverty of spirit and lack of
sincerity by rhetorical tricks or verbal cleverness. And if he
possesses the two essential requirements, the simplest language will
suffice.

These qualities are exemplified at their best by Turgenev and Tolstoy.
They both had a strong social consciousness; they both grappled with
the problems of human welfare; they were both artists in the larger
sense, that is, in their truthful representation of life. Turgenev was
an artist also in the narrower sense--in a keen appreciation Of form.
Thoroughly Occidental in his tastes, he sought the regeneration of
Russia in radical progress along the lines of European democracy.
Tolstoy, on the other hand, sought the salvation of mankind in a
return to the primitive life and primitive Christian religion.

The very first work of importance by Turgenev, \emph{A Sportsman's
Sketches}, dealt with the question of serfdom, and it wielded
tremendous influence in bringing about its abolition. Almost every
succeeding book of his, from \emph{Rudin} through \emph{Fathers and Sons} to
\emph{Virgin Soil}, presented vivid pictures of contemporary Russian
society, with its problems, the clash of ideas between the old and the
new generations, and the struggles, the aspirations and the thoughts
that engrossed the advanced youth of Russia; so that his collected
works form a remarkable literary record of the successive movements of
Russian society in a period of preparation, fraught with epochal
significance, which culminated in the overthrow of Czarism and the
inauguration of a new and true democracy, marking the beginning,
perhaps, of a radical transformation the world over.

``The greatest writer of Russia." That is Turgenev's estimate of
Tolstoy. ``A second Shakespeare!" was Flaubert's enthusiastic outburst.
The Frenchman's comparison is not wholly illuminating. The one point
of resemblance between the two authors is simply in the tremendous
magnitude of their genius. Each is a Colossus. Each creates a whole
world of characters, from kings and princes and ladies to servants and
maids and peasants. But how vastly divergent the angle of approach!
Anna Karenina may have all the subtle womanly charm of an Olivia or a
Portia, but how different her trials. Shakespeare could not have
treated Anna's problems at all. Anna could not have appeared in his
pages except as a sinning Gertrude, the mother of Hamlet. Shakespeare
had all the prejudices of his age. He accepted the world as it is with
its absurd moralities, its conventions and institutions and social
classes. A gravedigger is naturally inferior to a lord, and if he is
to be presented at all, he must come on as a clown. The people are
always a mob, the rabble. Tolstoy, is the revolutionist, the
iconoclast. He has the completest independence of mind. He utterly
refuses to accept established opinions just because they are
established. He probes into the right and wrong of things. His is a
broad, generous universal democracy, his is a comprehensive sympathy,
his an absolute incapacity to evaluate human beings according to
station, rank or profession, or any standard but that of spiritual
worth. In all this he was a complete contrast to Shakespeare. Each of
the two men was like a creature of a higher world, possessed of
supernatural endowments. Their omniscience of all things human, their
insight into the hiddenmost springs of men's actions appear
miraculous. But Shakespeare makes the impression of detachment from
his works. The works do not reveal the man; while in Tolstoy the
greatness of the man blends with the greatness of the genius. Tolstoy
was no mere oracle uttering profundities he wot not of. As the social,
religious and moral tracts that he wrote in the latter period of his
life are instinct with a literary beauty of which he never could
divest himself, and which gave an artistic value even to his sermons,
so his earlier novels show a profound concern for the welfare of
society, a broad, humanitarian spirit, a bigness of soul that included
prince and pauper alike.

Is this extravagant praise? Then let me echo William Dean Howells: ``I
know very well that I do not speak of Tolstoy's books in measured
terms; I cannot."

The Russian writers so far considered have made valuable contributions
to the short story; but, with the exception of Pushkin, whose
reputation rests chiefly upon his poetry, their best work, generally,
was in the field of the long novel. It was the novel that gave Russian
literature its pre-eminence. It could not have been otherwise, since
Russia is young as a literary nation, and did not come of age until
the period at which the novel was almost the only form of literature
that counted. If, therefore, Russia was to gain distinction in the
world of letters, it could be only through the novel. Of the measure
of her success there is perhaps no better testimony than the words of
Matthew Arnold, a critic certainly not given to overstatement. ``The
Russian novel," he wrote in 1887, ``has now the vogue, and deserves to
have it... The Russian novelist is master of a spell to which the
secret of human nature--both what is external and internal, gesture
and manner no less than thought and feeling--willingly make themselves
known... In that form of imaginative literature, which in our day is
the most popular and the most possible, the Russians at the present
moment seem to me to hold the field."

With the strict censorship imposed on Russian writers, many of them
who might perhaps have contented themselves with expressing their
opinions in essays, were driven to conceal their meaning under the
guise of satire or allegory; which gave rise to a peculiar genre of
literature, a sort of editorial or essay done into fiction, in which
the satirist Saltykov, a contemporary of Turgenev and Dostoyevsky, who
wrote under the pseudonym of Shchedrin, achieved the greatest success
and popularity.

It was not however, until the concluding quarter of the last century
that writers like Korolenko and Garshin arose, who devoted themselves
chiefly to the cultivation of the short story. With Anton Chekhov the
short story assumed a position of importance alongside the larger
works of the great Russian masters. Gorky and Andreyev made the short
story do the same service for the active revolutionary period in the
last decade of the nineteenth century down to its temporary defeat in
1906 that Turgenev rendered in his series of larger novels for the
period of preparation. But very different was the voice of Gorky, the
man sprung from the people, the embodiment of all the accumulated
wrath and indignation of centuries of social wrong and oppression,
from the gentlemanly tones of the cultured artist Turgenev. Like a
mighty hammer his blows fell upon the decaying fabric of the old
society. His was no longer a feeble, despairing protest. With the
strength and confidence of victory he made onslaught upon onslaught on
the old institutions until they shook and almost tumbled. And when
reaction celebrated its short-lived triumph and gloom settled again
upon his country and most of his co-fighters withdrew from the battle
in despair, some returning to the old-time Russian mood of
hopelessness, passivity and apathy, and some even backsliding into
wild orgies of literary debauchery, Gorky never wavered, never lost
his faith and hope, never for a moment was untrue to his principles.
Now, with the revolution victorious, he has come into his right, one
of the most respected, beloved and picturesque figures in the Russian
democracy.

Kuprin, the most facile and talented short-story writer next to
Chekhov, has, on the whole, kept well to the best literary traditions
of Russia, though he has frequently wandered off to extravagant sex
themes, for which he seems to display as great a fondness as
Artzybashev. Semyonov is a unique character in Russian literature, a
peasant who had scarcely mastered the most elementary mechanics of
writing when he penned his first story. But that story pleased
Tolstoy, who befriended and encouraged him. His tales deal altogether
with peasant life in country and city, and have a lifelikeness, an
artlessness, a simplicity striking even in a Russian author.

There is a small group of writers detached from the main current of
Russian literature who worship at the shrine of beauty and mysticism.
Of these Sologub has attained the highest reputation.

Rich as Russia has become in the short story, Anton Chekhov still
stands out as the supreme master, one of the greatest short-story
writers of the world. He was born in Taganarok, in the Ukraine, in
1860, the son of a peasant serf who succeeded in buying his freedom.
Anton Chekhov studied medicine, but devoted himself largely to
writing, in which, he acknowledged, his scientific training was of
great service. Though he lived only forty-four years, dying of
tuberculosis in 1904, his collected works consist of sixteen
fair-sized volumes of short stories, and several dramas besides. A few
volumes of his works have already appeared in English translation.

Critics, among them Tolstoy, have often compared Chekhov to
Maupassant. I find it hard to discover the resemblance. Maupassant
holds a supreme position as a short-story writer; so does Chekhov. But
there, it seems to me, the likeness ends.

The chill wind that blows from the atmosphere created by the
Frenchman's objective artistry is by the Russian commingled with the
warm breath of a great human sympathy. Maupassant never tells where
his sympathies lie, and you don't know; you only guess. Chekhov does
not tell you where his sympathies lie, either, but you know all the
same; you don't have to guess. And yet Chekhov is as objective as
Maupassant. In the chronicling of facts, conditions, and situations,
in the reproduction of characters, he is scrupulously true, hard, and
inexorable. But without obtruding his personality, he somehow manages
to let you know that he is always present, always at hand. If you
laugh, he is there to laugh with you; if you cry, he is there to shed
a tear with you; if you are horrified, he is horrified, too. It is a
subtle art by which he contrives to make one feel the nearness of
himself for all his objectiveness, so subtle that it defies analysis.
And yet it constitutes one of the great charms of his tales.

Chekhov's works show an astounding resourcefulness and versatility.
There is no monotony, no repetition. Neither in incident nor in
character are any two stories alike. The range of Chekhov's knowledge
of men and things seems to be unlimited, and he is extravagant in the
use of it. Some great idea which many a writer would consider
sufficient to expand into a whole novel he disposes of in a story of a
few pages. Take, for example, \emph{Vanka}, apparently but a mere episode
in the childhood of a nine-year-old boy; while it is really the
tragedy of a whole life in its tempting glimpses into a past
environment and ominous forebodings of the future--all contracted into
the space of four or five pages. Chekhov is lavish with his
inventiveness. Apparently, it cost him no effort to invent.

I have used the word inventiveness for lack of a better name. It
expresses but lamely the peculiar faculty that distinguishes Chekhov.
Chekhov does not really invent. He reveals. He reveals things that no
author before him has revealed. It is as though he possessed a special
organ which enabled him to see, hear and feel things of which we other
mortals did not even dream the existence. Yet when he lays them bare
we know that they are not fictitious, not invented, but as real as the
ordinary familiar facts of life. This faculty of his playing on all
conceivable objects, all conceivable emotions, no matter how
microscopic, endows them with life and a soul. By virtue of this power
\emph{The Steppe}, an uneventful record of peasants travelling day after
day through flat, monotonous fields, becomes instinct with dramatic
interest, and its 125 pages seem all too short. And by virtue of the
same attribute we follow with breathless suspense the minute
description of the declining days of a great scientist, who feels his
physical and mental faculties gradually ebbing away. \emph{A Tiresome
Story}, Chekhov calls it; and so it would be without the vitality
conjured into it by the magic touch of this strange genius.

Divination is perhaps a better term than invention. Chekhov divines
the most secret impulses of the soul, scents out what is buried in the
subconscious, and brings it up to the surface. Most writers are
specialists. They know certain strata of society, and when they
venture beyond, their step becomes uncertain. Chekhov's material is
only delimited by humanity. He is equally at home everywhere. The
peasant, the labourer, the merchant, the priest, the professional man,
the scholar, the military officer, and the government functionary,
Gentile or Jew, man, woman, or child--Chekhov is intimate with all of
them. His characters are sharply defined individuals, not types. In
almost all his stories, however short, the men and women and children
who play a part in them come out as clear, distinct personalities.
Ariadne is as vivid a character as Lilly, the heroine of Sudermann's
\emph{Song of Songs}; yet \emph{Ariadne} is but a single story in a volume of
stories. Who that has read \emph{The Darling} can ever forget her--the
woman who had no separate existence of her own, but thought the
thoughts, felt the feelings, and spoke the words of the men she loved?
And when there was no man to love any more, she was utterly crushed
until she found a child to take care of and to love; and then she sank
her personality in the boy as she had sunk it before in her husbands
and lover, became a mere reflection of him, and was happy again.

In the compilation of this volume I have been guided by the desire to
give the largest possible representation to the prominent authors of
the Russian short story, and to present specimens characteristic of
each. At the same time the element of interest has been kept in mind;
and in a few instances, as in the case of Korolenko, the selection of
the story was made with a view to its intrinsic merit and striking
qualities rather than as typifying the writer's art. It was, of
course, impossible in the space of one book to exhaust all that is
best. But to my knowledge, the present volume is the most
comprehensive anthology of the Russian short story in the English
language, and gives a fair notion of the achievement in that field.
All who enjoy good reading, I have no reason to doubt, will get
pleasure from it, and if, in addition, it will prove of assistance to
American students of Russian literature, I shall feel that the task
has been doubly worth the while.

Korolenko's \emph{Shades} and Andreyev's \emph{Lazarus} first appeared in
\emph{Current Opinion}, and Artzybashev's \emph{The Revolutionist} in the
\emph{Metropolitan Magazine}. I take pleasure in thanking Mr. Edward J.
Wheeler, editor of \emph{Current Opinion}, and Mr. Carl Hovey, editor of
the \emph{Metropolitan Magazine}, for permission to reprint them.

\medskip

\begin{center}
[Signature: Thomas Seltzer]
\end{center}

\medskip

\begin{quote}
    ``Everything is subordinated to two main requirements -- humanitarian
ideals and fidelity to life. This is the secret of the marvellous
simplicity of Russian literary art." -- THOMAS SELTZER.
\end{quote}
