\chapter{\textsc{The Christmas Tree And The Wedding}\\
\small \hspace{20pt}
by Fiodor M. Dostoyevsky}


\lettrine[lines=3,lhang=0.11,lraise=0,loversize=0.05]{T}{}%
he other day I saw a wedding... But no! I would rather tell you about
a Christmas tree. The wedding was superb. I liked it immensely. But
the other incident was still finer. I don't know why it is that the
sight of the wedding reminded me of the Christmas tree. This is the
way it happened:

Exactly five years ago, on New Year's Eve, I was invited to a
children's ball by a man high up in the business world, who had his
connections, his circle of acquaintances, and his intrigues. So it
seemed as though the children's ball was merely a pretext for the
parents to come together and discuss matters of interest to
themselves, quite innocently and casually.

I was an outsider, and, as I had no special matters to air, I was able
to spend the evening independently of the others. There was another
gentleman present who like myself had just stumbled upon this affair
of domestic bliss. He was the first to attract my attention. His
appearance was not that of a man of birth or high family. He was tall,
rather thin, very serious, and well dressed. Apparently he had no
heart for the family festivities. The instant he went off into a
corner by himself the smile disappeared from his face, and his thick
dark brows knitted into a frown. He knew no one except the host and
showed every sign of being bored to death, though bravely sustaining
the role of thorough enjoyment to the end. Later I learned that he was
a provincial, had come to the capital on some important, brain-racking
business, had brought a letter of recommendation to our host, and our
host had taken him under his protection, not at all \emph{con amore}. It
was merely out of politeness that he had invited him to the children's
ball.

They did not play cards with him, they did not offer him cigars. No
one entered into conversation with him. Possibly they recognised the
bird by its feathers from a distance. Thus, my gentleman, not knowing
what to do with his hands, was compelled to spend the evening stroking
his whiskers. His whiskers were really fine, but he stroked them so
assiduously that one got the feeling that the whiskers had come into
the world first and afterwards the man in order to stroke them.

There was another guest who interested me. But he was of quite a
different order. He was a personage. They called him Julian
Mastakovich. At first glance one could tell he was an honoured guest
and stood in the same relation to the host as the host to the
gentleman of the whiskers. The host and hostess said no end of amiable
things to him, were most attentive, wining him, hovering over him,
bringing guests up to be introduced, but never leading him to any one
else. I noticed tears glisten in our host's eyes when Julian
Mastakovich remarked that he had rarely spent such a pleasant evening.
Somehow I began to feel uncomfortable in this personage's presence.
So, after amusing myself with the children, five of whom, remarkably
well-fed young persons, were our host's, I went into a little
sitting-room, entirely unoccupied, and seated myself at the end that
was a conservatory and took up almost half the room.

The children were charming. They absolutely refused to resemble their
elders, notwithstanding the efforts of mothers and governesses. In a
jiffy they had denuded the Christmas tree down to the very last sweet
and had already succeeded in breaking half of their playthings before
they even found out which belonged to whom.

One of them was a particularly handsome little lad, dark-eyed,
curly-haired, who stubbornly persisted in aiming at me with his wooden
gun. But the child that attracted the greatest attention was his
sister, a girl of about eleven, lovely as a Cupid. She was quiet and
thoughtful, with large, full, dreamy eyes. The children had somehow
offended her, and she left them and walked into the same room that I
had withdrawn into. There she seated herself with her doll in a
corner.

"Her father is an immensely wealthy business man," the guests informed
each other in tones of awe. "Three hundred thousand rubles set aside
for her dowry already."

As I turned to look at the group from which I heard this news item
issuing, my glance met Julian Mastakovich's. He stood listening to the
insipid chatter in an attitude of concentrated attention, with his
hands behind his back and his head inclined to one side.

All the while I was quite lost in admiration of the shrewdness our
host displayed in the dispensing of the gifts. The little maid of the
many-rubied dowry received the handsomest doll, and the rest of the
gifts were graded in value according to the diminishing scale of the
parents' stations in life. The last child, a tiny chap of ten, thin,
red-haired, freckled, came into possession of a small book of nature
stories without illustrations or even head and tail pieces. He was the
governess's child. She was a poor widow, and her little boy, clad in a
sorry-looking little nankeen jacket, looked thoroughly crushed and
intimidated. He took the book of nature stories and circled slowly
about the children's toys. He would have given anything to play with
them. But he did not dare to. You could tell he already knew his
place.

I like to observe children. It is fascinating to watch the
individuality in them struggling for self-assertion. I could see that
the other children's things had tremendous charm for the red-haired
boy, especially a toy theatre, in which he was so anxious to take a
part that he resolved to fawn upon the other children. He smiled and
began to play with them. His one and only apple he handed over to a
puffy urchin whose pockets were already crammed with sweets, and he
even carried another youngster pickaback--all simply that he might be
allowed to stay with the theatre.

But in a few moments an impudent young person fell on him and gave him
a pummelling. He did not dare even to cry. The governess came and told
him to leave off interfering with the other children's games, and he
crept away to the same room the little girl and I were in. She let him
sit down beside her, and the two set themselves busily dressing the
expensive doll.

Almost half an hour passed, and I was nearly dozing off, as I sat
there in the conservatory half listening to the chatter of the
red-haired boy and the dowered beauty, when Julian Mastakovich entered
suddenly. He had slipped out of the drawing-room under cover of a
noisy scene among the children. From my secluded corner it had not
escaped my notice that a few moments before he had been eagerly
conversing with the rich girl's father, to whom he had only just been
introduced.

He stood still for a while reflecting and mumbling to himself, as if
counting something on his fingers.

"Three hundred--three hundred--eleven--twelve--thirteen--sixteen--in
five years! Let's say four per cent--five times twelve--sixty, and on
these sixty---. Let us assume that in five years it will amount
to--well, four hundred. Hm--hm! But the shrewd old fox isn't likely to
be satisfied with four per cent. He gets eight or even ten, perhaps.
Let's suppose five hundred, five hundred thousand, at least, that's
sure. Anything above that for pocket money--hm--"

He blew his nose and was about to leave the room when he spied the
girl and stood still. I, behind the plants, escaped his notice. He
seemed to me to be quivering with excitement. It must have been his
calculations that upset him so. He rubbed his hands and danced from
place to place, and kept getting more and more excited. Finally,
however, he conquered his emotions and came to a standstill. He cast a
determined look at the future bride and wanted to move toward her, but
glanced about first. Then, as if with a guilty conscience, he stepped
over to the child on tip-toe, smiling, and bent down and kissed her
head.

His coming was so unexpected that she uttered a shriek of alarm.

"What are you doing here, dear child?" he whispered, looking around
and pinching her cheek.

"We're playing."

"What, with him?" said Julian Mastakovich with a look askance at the
governess's child. "You should go into the drawing-room, my lad," he
said to him.

The boy remained silent and looked up at the man with wide-open eyes.
Julian Mastakovich glanced round again cautiously and bent down over
the girl.

"What have you got, a doll, my dear?"

"Yes, sir." The child quailed a little, and her brow wrinkled.

"A doll? And do you know, my dear, what dolls are made of?"

"No, sir," she said weakly, and lowered her head.

"Out of rags, my dear. You, boy, you go back to the drawing-room, to
the children," said Julian Mastakovich looking at the boy sternly.

The two children frowned. They caught hold of each other and would not
part.

"And do you know why they gave you the doll?" asked Julian
Mastakovich, dropping his voice lower and lower.

"No."

"Because you were a good, very good little girl the whole week."

Saying which, Julian Mastakovich was seized with a paroxysm of
agitation. He looked round and said in a tone faint, almost inaudible
with excitement and impatience:

"If I come to visit your parents will you love me, my dear?"

He tried to kiss the sweet little creature, but the red-haired boy saw
that she was on the verge of tears, and he caught her hand and sobbed
out loud in sympathy. That enraged the man.

"Go away! Go away! Go back to the other room, to your playmates."

"I don't want him to. I don't want him to! You go away!" cried the
girl. "Let him alone! Let him alone!" She was almost weeping.

There was a sound of footsteps in the doorway. Julian Mastakovich
started and straightened up his respectable body. The red-haired boy
was even more alarmed. He let go the girl's hand, sidled along the
wall, and escaped through the drawing-room into the dining-room.

Not to attract attention, Julian Mastakovich also made for the
dining-room. He was red as a lobster. The sight of himself in a mirror
seemed to embarrass him. Presumably he was annoyed at his own ardour
and impatience. Without due respect to his importance and dignity, his
calculations had lured and pricked him to the greedy eagerness of a
boy, who makes straight for his object--though this was not as yet an
object; it only would be so in five years' time. I followed the worthy
man into the dining-room, where I witnessed a remarkable play.

Julian Mastakovich, all flushed with vexation, venom in his look,
began to threaten the red-haired boy. The red-haired boy retreated
farther and farther until there was no place left for him to retreat
to, and he did not know where to turn in his fright.

"Get out of here! What are you doing here? Get out, I say, you
good-for-nothing! Stealing fruit, are you? Oh, so, stealing fruit! Get
out, you freckle face, go to your likes!"

The frightened child, as a last desperate resort, crawled quickly
under the table. His persecutor, completely infuriated, pulled out his
large linen handkerchief and used it as a lash to drive the boy out of
his position.

Here I must remark that Julian Mastakovich was a somewhat corpulent
man, heavy, well-fed, puffy-cheeked, with a paunch and ankles as round
as nuts. He perspired and puffed and panted. So strong was his dislike
(or was it jealousy?) of the child that he actually began to carry on
like a madman.

I laughed heartily. Julian Mastakovich turned. He was utterly confused
and for a moment, apparently, quite oblivious of his immense
importance. At that moment our host appeared in the doorway opposite.
The boy crawled out from under the table and wiped his knees and
elbows. Julian Mastakovich hastened to carry his handkerchief, which
he had been dangling by the corner, to his nose. Our host looked at
the three of us rather suspiciously. But, like a man who knows the
world and can readily adjust himself, he seized upon the opportunity
to lay hold of his very valuable guest and get what he wanted out of
him.

"Here's the boy I was talking to you about," he said, indicating the
red-haired child. "I took the liberty of presuming on your goodness in
his behalf."

"Oh," replied Julian Mastakovich, still not quite master of himself.

"He's my governess's son," our host continued in a beseeching tone.
"She's a poor creature, the widow of an honest official. That's why,
if it were possible for you--"

"Impossible, impossible!" Julian Mastakovich cried hastily. "You must
excuse me, Philip Alexeyevich, I really cannot. I've made inquiries.
There are no vacancies, and there is a waiting list of ten who have a
greater right--I'm sorry."

"Too bad," said our host. "He's a quiet, unobtrusive child."

"A very naughty little rascal, I should say," said Julian Mastakovich,
wryly. "Go away, boy. Why are you here still? Be off with you to the
other children."

Unable to control himself, he gave me a sidelong glance. Nor could I
control myself. I laughed straight in his face. He turned away and
asked our host, in tones quite audible to me, who that odd young
fellow was. They whispered to each other and left the room,
disregarding me.

I shook with laughter. Then I, too, went to the drawing-room. There
the great man, already surrounded by the fathers and mothers and the
host and the hostess, had begun to talk eagerly with a lady to whom he
had just been introduced. The lady held the rich little girl's hand.
Julian Mastakovich went into fulsome praise of her. He waxed ecstatic
over the dear child's beauty, her talents, her grace, her excellent
breeding, plainly laying himself out to flatter the mother, who
listened scarcely able to restrain tears of joy, while the father
showed his delight by a gratified smile.

The joy was contagious. Everybody shared in it. Even the children were
obliged to stop playing so as not to disturb the conversation. The
atmosphere was surcharged with awe. I heard the mother of the
important little girl, touched to her profoundest depths, ask Julian
Mastakovich in the choicest language of courtesy, whether he would
honour them by coming to see them. I heard Julian Mastakovich accept
the invitation with unfeigned enthusiasm. Then the guests scattered
decorously to different parts of the room, and I heard them, with
veneration in their tones, extol the business man, the business man's
wife, the business man's daughter, and, especially, Julian
Mastakovich.

"Is he married?" I asked out loud of an acquaintance of mine standing
beside Julian Mastakovich.

Julian Mastakovich gave me a venomous look.

"No," answered my acquaintance, profoundly shocked by
my--intentional--indiscretion.

\begin{center}
    \begin{verbatim}
       *       *       *       *       *
    \end{verbatim}
\end{center}

Not long ago I passed the Church of---. I was struck by the concourse
of people gathered there to witness a wedding. It was a dreary day. A
drizzling rain was beginning to come down. I made my way through the
throng into the church. The bridegroom was a round, well-fed,
pot-bellied little man, very much dressed up. He ran and fussed about
and gave orders and arranged things. Finally word was passed that the
bride was coming. I pushed through the crowd, and I beheld a
marvellous beauty whose first spring was scarcely commencing. But the
beauty was pale and sad. She looked distracted. It seemed to me even
that her eyes were red from recent weeping. The classic severity of
every line of her face imparted a peculiar significance and solemnity
to her beauty. But through that severity and solemnity, through the
sadness, shone the innocence of a child. There was something
inexpressibly naïve, unsettled and young in her features, which,
without words, seemed to plead for mercy.

They said she was just sixteen years old. I looked at the bridegroom
carefully. Suddenly I recognised Julian Mastakovich, whom I had not
seen again in all those five years. Then I looked at the bride
again.--Good God! I made my way, as quickly as I could, out of the
church. I heard gossiping in the crowd about the bride's wealth--about
her dowry of five hundred thousand rubles--so and so much for pocket
money.

"Then his calculations were correct," I thought, as I pressed out into
the street.
