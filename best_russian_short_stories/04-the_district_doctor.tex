\chapter{\textsc{The District Doctor}\\
\small \hspace{20pt}
by Ivan S. Turgenev}

\lettrine[lines=3,lhang=0.11,lraise=0,loversize=0.05]{O}{}%
ne day in autumn on my way back from a remote part of the country I
caught cold and fell ill. Fortunately the fever attacked me in the
district town at the inn; I sent for the doctor. In half-an-hour the
district doctor appeared, a thin, dark-haired man of middle height. He
prescribed me the usual sudorific, ordered a mustard-plaster to be put
on, very deftly slid a five-ruble note up his sleeve, coughing drily
and looking away as he did so, and then was getting up to go home, but
somehow fell into talk and remained. I was exhausted with
feverishness; I foresaw a sleepless night, and was glad of a little
chat with a pleasant companion. Tea was served. My doctor began to
converse freely. He was a sensible fellow, and expressed himself with
vigour and some humour. Queer things happen in the world: you may live
a long while with some people, and be on friendly terms with them, and
never once speak openly with them from your soul; with others you have
scarcely time to get acquainted, and all at once you are pouring out
to him--or he to you--all your secrets, as though you were at
confession. I don't know how I gained the confidence of my new
friend--anyway, with nothing to lead up to it, he told me a rather
curious incident; and here I will report his tale for the information
of the indulgent reader. I will try to tell it in the doctor's own
words.

"You don't happen to know," he began in a weak and quavering voice
(the common result of the use of unmixed Berezov snuff); "you don't
happen to know the judge here, Mylov, Pavel Lukich?... You don't know
him?... Well, it's all the same." (He cleared his throat and rubbed
his eyes.) "Well, you see, the thing happened, to tell you exactly
without mistake, in Lent, at the very time of the thaws. I was sitting
at his house--our judge's, you know--playing preference. Our judge is
a good fellow, and fond of playing preference. Suddenly" (the doctor
made frequent use of this word, suddenly) "they tell me, 'There's a
servant asking for you.' I say, 'What does he want?' They say, He has
brought a note--it must be from a patient.' 'Give me the note,' I say.
So it is from a patient--well and good--you understand--it's our bread
and butter... But this is how it was: a lady, a widow, writes to me;
she says, 'My daughter is dying. Come, for God's sake!' she says, 'and
the horses have been sent for you.'... Well, that's all right. But she
was twenty miles from the town, and it was midnight out of doors, and
the roads in such a state, my word! And as she was poor herself, one
could not expect more than two silver rubles, and even that
problematic; and perhaps it might only be a matter of a roll of linen
and a sack of oatmeal in \emph{payment}. However, duty, you know, before
everything: a fellow-creature may be dying. I hand over my cards at
once to Kalliopin, the member of the provincial commission, and return
home. I look; a wretched little trap was standing at the steps, with
peasant's horses, fat--too fat--and their coat as shaggy as felt; and
the coachman sitting with his cap off out of respect. Well, I think to
myself, 'It's clear, my friend, these patients aren't rolling in
riches.'... You smile; but I tell you, a poor man like me has to take
everything into consideration... If the coachman sits like a prince,
and doesn't touch his cap, and even sneers at you behind his beard,
and flicks his whip--then you may bet on six rubles. But this case, I
saw, had a very different air. However, I think there's no help for
it; duty before everything. I snatch up the most necessary drugs, and
set off. Will you believe it? I only just managed to get there at all.
The road was infernal: streams, snow, watercourses, and the dyke had
suddenly burst there--that was the worst of it! However, I arrived at
last. It was a little thatched house. There was a light in the
windows; that meant they expected me. I was met by an old lady, very
venerable, in a cap. 'Save her!' she says; 'she is dying.' I say,
'Pray don't distress yourself--Where is the invalid?' 'Come this way.'
I see a clean little room, a lamp in the corner; on the bed a girl of
twenty, unconscious. She was in a burning heat, and breathing
heavily--it was fever. There were two other girls, her sisters, scared
and in tears. 'Yesterday,' they tell me, 'she was perfectly well and
had a good appetite; this morning she complained of her head, and this
evening, suddenly, you see, like this.' I say again: 'Pray don't be
uneasy.' It's a doctor's duty, you know--and I went up to her and bled
her, told them to put on a mustard-plaster, and prescribed a mixture.
Meantime I looked at her; I looked at her, you know--there, by God! I
had never seen such a face!--she was a beauty, in a word! I felt quite
shaken with pity. Such lovely features; such eyes!... But, thank God!
she became easier; she fell into a perspiration, seemed to come to her
senses, looked round, smiled, and passed her hand over her face... Her
sisters bent over her. They ask, 'How are you?' 'All right,' she says,
and turns away. I looked at her; she had fallen asleep. 'Well,' I say,
'now the patient should be left alone.' So we all went out on tiptoe;
only a maid remained, in case she was wanted. In the parlour there was
a samovar standing on the table, and a bottle of rum; in our
profession one can't get on without it. They gave me tea; asked me to
stop the night... I consented: where could I go, indeed, at that time
of night? The old lady kept groaning. 'What is it?' I say; 'she will
live; don't worry yourself; you had better take a little rest
yourself; it is about two o'clock.' 'But will you send to wake me if
anything happens?' 'Yes, yes.' The old lady went away, and the girls
too went to their own room; they made up a bed for me in the parlour.
Well, I went to bed--but I could not get to sleep, for a wonder! for
in reality I was very tired. I could not get my patient out of my
head. At last I could not put up with it any longer; I got up
suddenly; I think to myself, 'I will go and see how the patient is
getting on.' Her bedroom was next to the parlour. Well, I got up, and
gently opened the door--how my heart beat! I looked in: the servant
was asleep, her mouth wide open, and even snoring, the wretch! but the
patient lay with her face towards me and her arms flung wide apart,
poor girl! I went up to her ... when suddenly she opened her eyes and
stared at me! 'Who is it? who is it?' I was in confusion. 'Don't be
alarmed, madam,' I say; 'I am the doctor; I have come to see how you
feel.' 'You the doctor?' 'Yes, the doctor; your mother sent for me
from the town; we have bled you, madam; now pray go to sleep, and in a
day or two, please God! we will set you on your feet again.' 'Ah, yes,
yes, doctor, don't let me die... please, please.' 'Why do you talk
like that? God bless you!' She is in a fever again, I think to myself;
I felt her pulse; yes, she was feverish. She looked at me, and then
took me by the hand. 'I will tell you why I don't want to die: I will
tell you... Now we are alone; and only, please don't you ... not to
any one ... Listen...' I bent down; she moved her lips quite to my
ear; she touched my cheek with her hair--I confess my head went
round--and began to whisper... I could make out nothing of it... Ah,
she was delirious! ... She whispered and whispered, but so quickly,
and as if it were not in Russian; at last she finished, and shivering
dropped her head on the pillow, and threatened me with her finger:
'Remember, doctor, to no one.' I calmed her somehow, gave her
something to drink, waked the servant, and went away."

At this point the doctor again took snuff with exasperated energy, and
for a moment seemed stupefied by its effects.

"However," he continued, "the next day, contrary to my expectations,
the patient was no better. I thought and thought, and suddenly decided
to remain there, even though my other patients were expecting me...
And you know one can't afford to disregard that; one's practice
suffers if one does. But, in the first place, the patient was really
in danger; and secondly, to tell the truth, I felt strongly drawn to
her. Besides, I liked the whole family. Though they were really badly
off, they were singularly, I may say, cultivated people... Their
father had been a learned man, an author; he died, of course, in
poverty, but he had managed before he died to give his children an
excellent education; he left a lot of books too. Either because I
looked after the invalid very carefully, or for some other reason;
anyway, I can venture to say all the household loved me as if I were
one of the family... Meantime the roads were in a worse state than
ever; all communications, so to say, were cut off completely; even
medicine could with difficulty be got from the town... The sick girl
was not getting better... Day after day, and day after day ... but ...
here..." (The doctor made a brief pause.) "I declare I don't know how
to tell you."... (He again took snuff, coughed, and swallowed a little
tea.) "I will tell you without beating about the bush. My patient ...
how should I say?... Well she had fallen in love with me ... or, no,
it was not that she was in love ... however ... really, how should one
say?" (The doctor looked down and grew red.) "No," he went on quickly,
"in love, indeed! A man should not over-estimate himself. She was an
educated girl, clever and well-read, and I had even forgotten my
Latin, one may say, completely. As to appearance" (the doctor looked
himself over with a smile) "I am nothing to boast of there either. But
God Almighty did not make me a fool; I don't take black for white; I
know a thing or two; I could see very clearly, for instance that
Aleksandra Andreyevna--that was her name--did not feel love for me,
but had a friendly, so to say, inclination--a respect or something for
me. Though she herself perhaps mistook this sentiment, anyway this was
her attitude; you may form your own judgment of it. But," added the
doctor, who had brought out all these disconnected sentences without
taking breath, and with obvious embarrassment, "I seem to be wandering
rather--you won't understand anything like this ... There, with your
leave, I will relate it all in order."

He drank off a glass of tea, and began in a calmer voice.

"Well, then. My patient kept getting worse and worse. You are not a
doctor, my good sir; you cannot understand what passes in a poor
fellow's heart, especially at first, when he begins to suspect that
the disease is getting the upper hand of him. What becomes of his
belief in himself? You suddenly grow so timid; it's indescribable. You
fancy then that you have forgotten everything you knew, and that the
patient has no faith in you, and that other people begin to notice how
distracted you are, and tell you the symptoms with reluctance; that
they are looking at you suspiciously, whispering... Ah! it's horrid!
There must be a remedy, you think, for this disease, if one could find
it. Isn't this it? You try--no, that's not it! You don't allow the
medicine the necessary time to do good... You clutch at one thing,
then at another. Sometimes you take up a book of medical
prescriptions--here it is, you think! Sometimes, by Jove, you pick one
out by chance, thinking to leave it to fate... But meantime a
fellow-creature's dying, and another doctor would have saved him. 'We
must have a consultation,' you say; 'I will not take the
responsibility on myself.' And what a fool you look at such times!
Well, in time you learn to bear it; it's nothing to you. A man has
died--but it's not your fault; you treated him by the rules. But
what's still more torture to you is to see blind faith in you, and to
feel yourself that you are not able to be of use. Well, it was just
this blind faith that the whole of Aleksandra Andreyevna's family had
in me; they had forgotten to think that their daughter was in danger.
I, too, on my side assure them that it's nothing, but meantime my
heart sinks into my boots. To add to our troubles, the roads were in
such a state that the coachman was gone for whole days together to get
medicine. And I never left the patient's room; I could not tear myself
away; I tell her amusing stories, you know, and play cards with her. I
watch by her side at night. The old mother thanks me with tears in her
eyes; but I think to myself, 'I don't deserve your gratitude.' I
frankly confess to you--there is no object in concealing it now--I was
in love with my patient. And Aleksandra Andreyevna had grown fond of
me; she would not sometimes let any one be in her room but me. She
began to talk to me, to ask me questions; where I had studied, how I
lived, who are my people, whom I go to see. I feel that she ought not
to talk; but to forbid her to--to forbid her resolutely, you know--I
could not. Sometimes I held my head in my hands, and asked myself,
"What are you doing, villain?"... And she would take my hand and hold
it, give me a long, long look, and turn away, sigh, and say, 'How good
you are!' Her hands were so feverish, her eyes so large and languid...
'Yes,' she says, 'you are a good, kind man; you are not like our
neighbours... No, you are not like that... Why did I not know you till
now!' 'Aleksandra Andreyevna, calm yourself,' I say... 'I feel,
believe me, I don't know how I have gained ... but there, calm
yourself... All will be right; you will be well again.' And meanwhile
I must tell you," continued the doctor, bending forward and raising
his eyebrows, "that they associated very little with the neighbours,
because the smaller people were not on their level, and pride hindered
them from being friendly with the rich. I tell you, they were an
exceptionally cultivated family; so you know it was gratifying for me.
She would only take her medicine from my hands ... she would lift
herself up, poor girl, with my aid, take it, and gaze at me... My
heart felt as if it were bursting. And meanwhile she was growing worse
and worse, worse and worse, all the time; she will die, I think to
myself; she must die. Believe me, I would sooner have gone to the
grave myself; and here were her mother and sisters watching me,
looking into my eyes ... and their faith in me was wearing away.
'Well? how is she?' 'Oh, all right, all right!' All right, indeed! My
mind was failing me. Well, I was sitting one night alone again by my
patient. The maid was sitting there too, and snoring away in full
swing; I can't find fault with the poor girl, though; she was worn out
too. Aleksandra Andreyevna had felt very unwell all the evening; she
was very feverish. Until midnight she kept tossing about; at last she
seemed to fall asleep; at least, she lay still without stirring. The
lamp was burning in the corner before the holy image. I sat there, you
know, with my head bent; I even dozed a little. Suddenly it seemed as
though some one touched me in the side; I turned round... Good God!
Aleksandra Andreyevna was gazing with intent eyes at me ... her lips
parted, her cheeks seemed burning. 'What is it?' 'Doctor, shall I
die?' 'Merciful Heavens!' 'No, doctor, no; please don't tell me I
shall live ... don't say so... If you knew... Listen! for God's sake
don't conceal my real position,' and her breath came so fast. 'If I
can know for certain that I must die ... then I will tell you all--
all!' 'Aleksandra Andreyevna, I beg!' 'Listen; I have not been asleep
at all ... I have been looking at you a long while... For God's
sake!... I believe in you; you are a good man, an honest man; I
entreat you by all that is sacred in the world--tell me the truth! If
you knew how important it is for me... Doctor, for God's sake tell
me... Am I in danger?' 'What can I tell you, Aleksandra Andreyevna,
pray?' 'For God's sake, I beseech you!' 'I can't disguise from you,' I
say, 'Aleksandra Andreyevna; you are certainly in danger; but God is
merciful.' 'I shall die, I shall die.' And it seemed as though she
were pleased; her face grew so bright; I was alarmed. 'Don't be
afraid, don't be afraid! I am not frightened of death at all.' She
suddenly sat up and leaned on her elbow. 'Now ... yes, now I can tell
you that I thank you with my whole heart ... that you are kind and
good--that I love you!' I stare at her, like one possessed; it was
terrible for me, you know. 'Do you hear, I love you!' 'Aleksandra
Andreyevna, how have I deserved--' 'No, no, you don't--you don't
understand me.'... And suddenly she stretched out her arms, and taking
my head in her hands, she kissed it... Believe me, I almost screamed
aloud... I threw myself on my knees, and buried my head in the pillow.
She did not speak; her fingers trembled in my hair; I listen; she is
weeping. I began to soothe her, to assure her... I really don't know
what I did say to her. 'You will wake up the girl,' I say to her;
'Aleksandra Andreyevna, I thank you ... believe me ... calm yourself.'
'Enough, enough!' she persisted; 'never mind all of them; let them
wake, then; let them come in--it does not matter; I am dying, you
see... And what do you fear? why are you afraid? Lift up your head...
Or, perhaps, you don't love me; perhaps I am wrong... In that case,
forgive me.' 'Aleksandra Andreyevna, what are you saying!... I love
you, Aleksandra Andreyevna.' She looked straight into my eyes, and
opened her arms wide. 'Then take me in your arms.' I tell you frankly,
I don't know how it was I did not go mad that night. I feel that my
patient is killing herself; I see that she is not fully herself; I
understand, too, that if she did not consider herself on the point of
death, she would never have thought of me; and, indeed, say what you
will, it's hard to die at twenty without having known love; this was
what was torturing her; this was why, in despair, she caught at
me--do you understand now? But she held me in her arms, and would not
let me go. 'Have pity on me, Aleksandra Andreyevna, and have pity on
yourself,' I say. 'Why,' she says; 'what is there to think of? You
know I must die.' ... This she repeated incessantly ... 'If I knew
that I should return to life, and be a proper young lady again, I
should be ashamed ... of course, ashamed ... but why now?' 'But who
has said you will die?' 'Oh, no, leave off! you will not deceive me;
you don't know how to lie--look at your face.' ... 'You shall live,
Aleksandra Andreyevna; I will cure you; we will ask your mother's
blessing ... we will be united--we will be happy.' 'No, no, I have
your word; I must die ... you have promised me ... you have told me.'
... It was cruel for me--cruel for many reasons. And see what trifling
things can do sometimes; it seems nothing at all, but it's painful. It
occurred to her to ask me, what is my name; not my surname, but my
first name. I must needs be so unlucky as to be called Trifon. Yes,
indeed; Trifon Ivanich. Every one in the house called me doctor.
However, there's no help for it. I say, 'Trifon, madam.' She frowned,
shook her head, and muttered something in French--ah, something
unpleasant, of course!--and then she laughed--disagreeably too. Well,
I spent the whole night with her in this way. Before morning I went
away, feeling as though I were mad. When I went again into her room it
was daytime, after morning tea. Good God! I could scarcely recognise
her; people are laid in their grave looking better than that. I swear
to you, on my honour, I don't understand--I absolutely don't
understand--now, how I lived through that experience. Three days and
nights my patient still lingered on. And what nights! What things she
said to me! And on the last night--only imagine to yourself--I was
sitting near her, and kept praying to God for one thing only: 'Take
her,' I said, 'quickly, and me with her.' Suddenly the old mother
comes unexpectedly into the room. I had already the evening before
told her---the mother--there was little hope, and it would be well to
send for a priest. When the sick girl saw her mother she said: 'It's
very well you have come; look at us, we love one another--we have
given each other our word.' 'What does she say, doctor? what does she
say?' I turned livid. 'She \emph{is} wandering,' I say; 'the fever.' But
she: 'Hush, hush; you told me something quite different just now, and
have taken my ring. Why do you pretend? My mother is good--she will
forgive--she will understand--and I am dying. ... I have no need to
tell lies; give me your hand.' I jumped up and ran out of the room.
The old lady, of course, guessed how it was.

"I will not, however, weary you any longer, and to me too, of course,
it's painful to recall all this. My patient passed away the next day.
God rest her soul!" the doctor added, speaking quickly and with a
sigh. "Before her death she asked her family to go out and leave me
alone with her."

"'Forgive me,' she said; 'I am perhaps to blame towards you ... my
illness ... but believe me, I have loved no one more than you ... do
not forget me ... keep my ring.'"

The doctor turned away; I took his hand.

"Ah!" he said, "let us talk of something else, or would you care to
play preference for a small stake? It is not for people like me to
give way to exalted emotions. There's only one thing for me to think
of; how to keep the children from crying and the wife from scolding.
Since then, you know, I have had time to enter into lawful wedlock, as
they say... Oh ... I took a merchant's daughter--seven thousand for
her dowry. Her name's Akulina; it goes well with Trifon. She is an
ill-tempered woman, I must tell you, but luckily she's asleep all
day... Well, shall it be preference?"

We sat down to preference for halfpenny points. Trifon Ivanich won two
rubles and a half from me, and went home late, well pleased with his
success.


