\chapter{\textsc{The Cloak}\\
\small \hspace{20pt} by Nikolay V. Gogol}


\lettrine[lines=3,lhang=0.11,lraise=0,loversize=0.05]{I}{}%
n the department of---, but it is better not to mention the
department. The touchiest things in the world are departments,
regiments, courts of justice, in a word, all branches of public
service. Each individual nowadays thinks all society insulted in his
person. Quite recently, a complaint was received from a district chief
of police in which he plainly demonstrated that all the imperial
institutions were going to the dogs, and that the Czar's sacred name
was being taken in vain; and in proof he appended to the complaint a
romance, in which the district chief of police is made to appear about
once in every ten pages, and sometimes in a downright drunken
condition. Therefore, in order to avoid all unpleasantness, it will be
better to designate the department in question, as a certain
department.

So, in a certain department there was a certain official--not a very
notable one, it must be allowed--short of stature, somewhat
pock-marked, red-haired, and mole-eyed, with a bald forehead, wrinkled
cheeks, and a complexion of the kind known as sanguine. The St.
Petersburg climate was responsible for this. As for his official
rank--with us Russians the rank comes first--he was what is called a
perpetual titular councillor, over which, as is well known, some
writers make merry and crack their jokes, obeying the praiseworthy
custom of attacking those who cannot bite back.

His family name was Bashmachkin. This name is evidently derived from
bashmak (shoe); but, when, at what time, and in what manner, is not
known. His father and grandfather, and all the Bashmachkins, always
wore boots, which were resoled two or three times a year. His name was
Akaky Akakiyevich. It may strike the reader as rather singular and
far-fetched; but he may rest assured that it was by no means
far-fetched, and that the circumstances were such that it would have
been impossible to give him any other.

This was how it came about.

Akaky Akakiyevich was born, if my memory fails me not, in the evening
on the 23rd of March. His mother, the wife of a Government official,
and a very fine woman, made all due arrangements for having the child
baptised. She was lying on the bed opposite the door; on her right
stood the godfather, Ivan Ivanovich Eroshkin, a most estimable man,
who served as the head clerk of the senate; and the godmother, Arina
Semyonovna Bielobrinshkova, the wife of an officer of the quarter, and
a woman of rare virtues. They offered the mother her choice of three
names, Mokiya, Sossiya, or that the child should be called after the
martyr Khozdazat. "No," said the good woman, "all those names are
poor." In order to please her, they opened the calendar at another
place; three more names appeared, Triphily, Dula, and Varakhasy. "This
is awful," said the old woman. "What names! I truly never heard the
like. I might have put up with Varadat or Varukh, but not Triphily and
Varakhasy!" They turned to another page and found Pavsikakhy and
Vakhtisy. "Now I see," said the old woman, "that it is plainly fate.
And since such is the case, it will be better to name him after his
father. His father's name was Akaky, so let his son's name be Akaky
too." In this manner he became Akaky Akakiyevich. They christened the
child, whereat he wept, and made a grimace, as though he foresaw that
he was to be a titular councillor.

In this manner did it all come about. We have mentioned it in order
that the reader might see for himself that it was a case of necessity,
and that it was utterly impossible to give him any other name.

When and how he entered the department, and who appointed him, no one
could remember. However much the directors and chiefs of all kinds
were changed, he was always to be seen in the same place, the same
attitude, the same occupation--always the letter-copying clerk--so
that it was afterwards affirmed that he had been born in uniform with
a bald head. No respect was shown him in the department. The porter
not only did not rise from his seat when he passed, but never even
glanced at him, any more than if a fly had flown through the
reception-room. His superiors treated him in coolly despotic fashion.
Some insignificant assistant to the head clerk would thrust a paper
under his nose without so much as saying, "Copy," or, "Here's an
interesting little case," or anything else agreeable, as is customary
amongst well-bred officials. And he took it, looking only at the
paper, and not observing who handed it to him, or whether he had the
right to do so; simply took it, and set about copying it.

The young officials laughed at and made fun of him, so far as their
official wit permitted; told in his presence various stories concocted
about him, and about his landlady, an old woman of seventy; declared
that she beat him; asked when the wedding was to be; and strewed bits
of paper over his head, calling them snow. But Akaky Akakiyevich
answered not a word, any more than if there had been no one there
besides himself. It even had no effect upon his work. Amid all these
annoyances he never made a single mistake in a letter. But if the
joking became wholly unbearable, as when they jogged his head, and
prevented his attending to his work, he would exclaim:

"Leave me alone! Why do you insult me?"

And there was something strange in the words and the voice in which
they were uttered. There was in it something which moved to pity; so
much so that one young man, a newcomer, who, taking pattern by the
others, had permitted himself to make sport of Akaky, suddenly stopped
short, as though all about him had undergone a transformation, and
presented itself in a different aspect. Some unseen force repelled him
from the comrades whose acquaintance he had made, on the supposition
that they were decent, well-bred men. Long afterwards, in his gayest
moments, there recurred to his mind the little official with the bald
forehead, with his heart-rending words, "Leave me alone! Why do you
insult me?" In these moving words, other words resounded--"I am thy
brother." And the young man covered his face with his hand; and many a
time afterwards, in the course of his life, shuddered at seeing how
much inhumanity there is in man, how much savage coarseness is
concealed beneath refined, cultured, worldly refinement, and even, O
God! in that man whom the world acknowledges as honourable and
upright.

It would be difficult to find another man who lived so entirely for
his duties. It is not enough to say that Akaky laboured with zeal; no,
he laboured with love. In his copying, he found a varied and agreeable
employment. Enjoyment was written on his face; some letters were even
favourites with him; and when he encountered these, he smiled, winked,
and worked with his lips, till it seemed as though each letter might
be read in his face, as his pen traced it. If his pay had been in
proportion to his zeal, he would, perhaps, to his great surprise, have
been made even a councillor of state. But he worked, as his
companions, the wits, put it, like a horse in a mill.

However, it would be untrue to say that no attention was paid to him.
One director being a kindly man, and desirous of rewarding him for his
long service, ordered him to be given something more important than
mere copying. So he was ordered to make a report of an already
concluded affair, to another department; the duty consisting simply in
changing the heading and altering a few words from the first to the
third person. This caused him so much toil, that he broke into a
perspiration, rubbed his forehead, and finally said, "No, give me
rather something to copy." After that they let him copy on forever.

Outside this copying, it appeared that nothing existed for him. He
gave no thought to his clothes. His uniform was not green, but a sort
of rusty-meal colour. The collar was low, so that his neck, in spite
of the fact that it was not long, seemed inordinately so as it emerged
from it, like the necks of the plaster cats which pedlars carry about
on their heads. And something was always sticking to his uniform,
either a bit of hay or some trifle. Moreover, he had a peculiar knack,
as he walked along the street, of arriving beneath a window just as
all sorts of rubbish was being flung out of it; hence he always bore
about on his hat scraps of melon rinds, and other such articles. Never
once in his life did he give heed to what was going on every day to
the street; while it is well known that his young brother officials
trained the range of their glances till they could see when any one's
trouser-straps came undone upon the opposite sidewalk, which always
brought a malicious smile to their faces. But Akaky Akakiyevich saw in
all things the clean, even strokes of his written lines; and only when
a horse thrust his nose, from some unknown quarter, over his shoulder,
and sent a whole gust of wind down his neck from his nostrils, did he
observe that he was not in the middle of a line, but in the middle of
the street.

On reaching home, he sat down at once at the table, sipped his
cabbage-soup up quickly, and swallowed a bit of beef with onions,
never noticing their taste, and gulping down everything with flies and
anything else which the Lord happened to send at the moment. When he
saw that his stomach was beginning to swell, he rose from the table,
and copied papers which he had brought home. If there happened to be
none, he took copies for himself, for his own gratification,
especially if the document was noteworthy, not on account of its
style, but of its being addressed to some distinguished person.

Even at the hour when the grey St. Petersburg sky had quite
disappeared, and all the official world had eaten or dined, each as he
could, in accordance with the salary he received and his own fancy;
when, all were resting from the department jar of pens, running to and
fro, for their own and other people's indispensable occupations, and
from all the work that an uneasy man makes willingly for himself,
rather than what is necessary; when, officials hasten to dedicate to
pleasure the time which is left to them, one bolder than the rest,
going to the theatre; another; into the street looking under the
bonnets; another, wasting his evening in compliments to some pretty
girl, the star of a small official circle; another--and this is the
common case of all--visiting his comrades on the third or fourth
floor, in two small rooms with an ante-room or kitchen, and some
pretensions to fashion, such as a lamp or some other trifle which has
cost many a sacrifice of dinner or pleasure trip; in a word, at the
hour when all officials disperse among the contracted quarters of
their friends, to play whist, as they sip their tea from glasses with
a kopek's worth of sugar, smoke long pipes, relate at time some bits
of gossip which a Russian man can never, under any circumstances,
refrain from, and when there is nothing else to talk of, repeat
eternal anecdotes about the commandant to whom they had sent word that
the tails of the horses on the Falconet Monument had been cut off;
when all strive to divert themselves, Akaky Akakiyevich indulged in no
kind of diversion. No one could even say that he had seen him at any
kind of evening party. Having written to his heart's content, he lay
down to sleep, smiling at the thought of the coming day--of what God
might send him to copy on the morrow.

Thus flowed on the peaceful life of the man, who, with a salary of
four hundred rubles, understood how to be content with his lot; and
thus it would have continued to flow on, perhaps, to extreme old age,
were it not that there are various ills strewn along the path of life
for titular councillors as well as for private, actual, court, and
every other species of councillor, even to those who never give any
advice or take any themselves.

There exists in St. Petersburg a powerful foe of all who receive a
salary of four hundred rubles a year, or there-abouts. This foe is no
other than the Northern cold, although it is said to be very healthy.
At nine o'clock in the morning, at the very hour when the streets are
filled with men bound for the various official departments, it begins
to bestow such powerful and piercing nips on all noses impartially,
that the poor officials really do not know what to do with them. At an
hour, when the foreheads of even those who occupy exalted positions
ache with the cold, and tears start to their eyes, the poor titular
councillors are sometimes quite unprotected. Their only salvation lies
in traversing as quickly as possible, in their thin little cloaks,
five or six streets, and then warming their feet in the porter's room,
and so thawing all their talents and qualifications for official
service, which had become frozen on the way.

Akaky Akakiyevich had felt for some time that his back and shoulders
were paining with peculiar poignancy, in spite of the fact that he
tried to traverse the distance with all possible speed. He began
finally to wonder whether the fault did not lie in his cloak. He
examined it thoroughly at home, and discovered that in two places,
namely, on the back and shoulders, it had become thin as gauze. The
cloth was worn to such a degree that he could see through it, and the
lining had fallen into pieces. You must know that Akaky Akakiyevich's
cloak served as an object of ridicule to the officials. They even
refused it the noble name of cloak, and called it a cape. In fact, it
was of singular make, its collar diminishing year by year to serve to
patch its other parts. The patching did not exhibit great skill on the
part of the tailor, and was, in fact, baggy and ugly. Seeing how the
matter stood, Akaky Akakiyevich decided that it would be necessary to
take the cloak to Petrovich, the tailor, who lived somewhere on the
fourth floor up a dark staircase, and who, in spite of his having but
one eye and pock-marks all over his face, busied himself with
considerable success in repairing the trousers and coats of officials
and others; that is to say, when he was sober and not nursing some
other scheme in his head.

It is not necessary to say much about this tailor, but as it is the
custom to have the character of each personage in a novel clearly
defined there is no help for it, so here is Petrovich the tailor. At
first he was called only Grigory, and was some gentleman's serf. He
commenced calling himself Petrovich from the time when he received his
free papers, and further began to drink heavily on all holidays, at
first on the great ones, and then on all church festivals without
discrimination, wherever a cross stood in the calendar. On this point
he was faithful to ancestral custom; and when quarrelling with his
wife, he called her a low female and a German. As we have mentioned
his wife, it will be necessary to say a word or two about her.
Unfortunately, little is known of her beyond the fact that Petrovich
had a wife, who wore a cap and a dress, but could not lay claim to
beauty, at least, no one but the soldiers of the guard even looked
under her cap when they met her.

Ascending the staircase which led to Petrovich's room--which staircase
was all soaked with dish-water and reeked with the smell of spirits
which affects the eyes, and is an inevitable adjunct to all dark
stairways in St. Petersburg houses--ascending the stairs, Akaky
Akakiyevich pondered how much Petrovich would ask, and mentally
resolved not to give more than two rubles. The door was open, for the
mistress, in cooking some fish, had raised such a smoke in the kitchen
that not even the beetles were visible. Akaky Akakiyevich passed
through the kitchen unperceived, even by the housewife, and at length
reached a room where he beheld Petrovich seated on a large unpainted
table, with his legs tucked under him like a Turkish pasha. His feet
were bare, after the fashion of tailors as they sit at work; and the
first thing which caught the eye was his thumb, with a deformed nail
thick and strong as a turtle's shell. About Petrovich's neck hung a
skein of silk and thread, and upon his knees lay some old garment. He
had been trying unsuccessfully for three minutes to thread his needle,
and was enraged at the darkness and even at the thread, growling in a
low voice, "It won't go through, the barbarian! you pricked me, you
rascal!"

Akaky Akakiyevich was vexed at arriving at the precise moment when
Petrovich was angry. He liked to order something of Petrovich when he
was a little downhearted, or, as his wife expressed it, "when he had
settled himself with brandy, the one-eyed devil!" Under such
circumstances Petrovich generally came down in his price very readily,
and even bowed and returned thanks. Afterwards, to be sure, his wife
would come, complaining that her husband had been drunk, and so had
fixed the price too low; but, if only a ten-kopek piece were added
then the matter would be settled. But now it appeared that Petrovich
was in a sober condition, and therefore rough, taciturn, and inclined
to demand, Satan only knows what price. Akaky Akakiyevich felt this,
and would gladly have beat a retreat, but he was in for it. Petrovich
screwed up his one eye very intently at him, and Akaky Akakiyevich
involuntarily said, "How do you do, Petrovich?"

"I wish you a good morning, sir," said Petrovich squinting at Akaky
Akakiyevich's hands, to see what sort of booty he had brought.

"Ah! I--to you, Petrovich, this--" It must be known that Akaky
Akakiyevich expressed himself chiefly by prepositions, adverbs, and
scraps of phrases which had no meaning whatever. If the matter was a
very difficult one, he had a habit of never completing his sentences,
so that frequently, having begun a phrase with the words, "This, in
fact, is quite--" he forgot to go on, thinking he had already finished
it.

"What is it?" asked Petrovich, and with his one eye scanned Akaky
Akakiyevich's whole uniform from the collar down to the cuffs, the
back, the tails and the button-holes, all of which were well known to
him, since they were his own handiwork. Such is the habit of tailors;
it is the first thing they do on meeting one.

"But I, here, this--Petrovich--a cloak, cloth--here you see,
everywhere, in different places, it is quite strong--it is a little
dusty and looks old, but it is new, only here in one place it is a
little--on the back, and here on one of the shoulders, it is a little
worn, yes, here on this shoulder it is a little--do you see? That is
all. And a little work--"

Petrovich took the cloak, spread it out, to begin with, on the table,
looked at it hard, shook his head, reached out his hand to the
window-sill for his snuff-box, adorned with the portrait of some
general, though what general is unknown, for the place where the face
should have been had been rubbed through by the finger and a square
bit of paper had been pasted over it. Having taken a pinch of snuff,
Petrovich held up the cloak, and inspected it against the light, and
again shook his head. Then he turned it, lining upwards, and shook his
head once more. After which he again lifted the general-adorned lid
with its bit of pasted paper, and having stuffed his nose with snuff,
dosed and put away the snuff-box, and said finally, "No, it is
impossible to mend it. It is a wretched garment!"

Akaky Akakiyevich's heart sank at these words.

"Why is it impossible, Petrovich?" he said, almost in the pleading
voice of a child. "All that ails it is, that it is worn on the
shoulders. You must have some pieces--"

"Yes, patches could be found, patches are easily found," said
Petrovich, "but there's nothing to sew them to. The thing is
completely rotten. If you put a needle to it--see, it will give way."

"Let it give way, and you can put on another patch at once."

"But there is nothing to put the patches on to. There's no use in
strengthening it. It is too far gone. It's lucky that it's cloth, for,
if the wind were to blow, it would fly away."

"Well, strengthen it again. How this, in fact--"

"No," said Petrovich decisively, "there is nothing to be done with it.
It's a thoroughly bad job. You'd better, when the cold winter weather
comes on, make yourself some gaiters out of it, because stockings are
not warm. The Germans invented them in order to make more money."
Petrovich loved on all occasions to have a fling at the Germans. "But
it is plain you must have a new cloak."

At the word "new" all grew dark before Akaky Akakiyevich's eyes, and
everything in the room began to whirl round. The only thing he saw
clearly was the general with the paper face on the lid of Petrovich's
snuff-box. "A new one?" said he, as if still in a dream. "Why, I have
no money for that."

"Yes, a new one," said Petrovich, with barbarous composure.

"Well, if it came to a new one, how--it--"

"You mean how much would it cost?"

"Yes."

"Well, you would have to lay out a hundred and fifty or more," said
Petrovich, and pursed up his lips significantly. He liked to produce
powerful effects, liked to stun utterly and suddenly, and then to
glance sideways to see what face the stunned person would put on the
matter.

"A hundred and fifty rubles for a cloak!" shrieked poor Akaky
Akakiyevich, perhaps for the first time in his life, for his voice had
always been distinguished for softness.

"Yes, sir," said Petrovich, "for any kind of cloak. If you have a
marten fur on the collar, or a silk-lined hood, it will mount up to
two hundred."

"Petrovich, please," said Akaky Akakiyevich in a beseeching tone, not
hearing, and not trying to hear, Petrovich's words, and disregarding
all his "effects," "some repairs, in order that it may wear yet a
little longer."

"No, it would only be a waste of time and money," said Petrovich. And
Akaky Akakiyevich went away after these words, utterly discouraged.
But Petrovich stood for some time after his departure, with
significantly compressed lips, and without betaking himself to his
work, satisfied that he would not be dropped, and an artistic tailor
employed.

Akaky Akakiyevich went out into the street as if in a dream. "Such an
affair!" he said to himself. "I did not think it had come to--" and
then after a pause, he added, "Well, so it is! see what it has come to
at last! and I never imagined that it was so!" Then followed a long
silence, after which he exclaimed, "Well, so it is! see what
already--nothing unexpected that--it would be nothing--what a strange
circumstance!" So saying, instead of going home, he went in exactly
the opposite direction without suspecting it. On the way, a
chimney-sweep bumped up against him, and blackened his shoulder, and a
whole hatful of rubbish landed on him from the top of a house which
was building. He did not notice it, and only when he ran against a
watchman, who, having planted his halberd beside him, was shaking some
snuff from his box into his horny hand, did he recover himself a
little, and that because the watchman said, "Why are you poking
yourself into a man's very face? Haven't you the pavement?" This
caused him to look about him, and turn towards home.

There only, he finally began to collect his thoughts, and to survey
his position in its clear and actual light, and to argue with himself,
sensibly and frankly, as with a reasonable friend, with whom one can
discuss private and personal matters. "No," said Akaky Akakiyevich,
"it is impossible to reason with Petrovich now. He is that--evidently,
his wife has been beating him. I'd better go to him on Sunday morning.
After Saturday night he will be a little cross-eyed and sleepy, for he
will want to get drunk, and his wife won't give him any money, and at
such a time, a ten-kopek piece in his hand will--he will become more
fit to reason with, and then the cloak and that--" Thus argued Akaky
Akakiyevich with himself regained his courage, and waited until the
first Sunday, when, seeing from afar that Petrovich's wife had left
the house, he went straight to him.

Petrovich's eye was indeed very much askew after Saturday. His head
drooped, and he was very sleepy; but for all that, as soon as he knew
what it was a question of, it seemed as though Satan jogged his
memory. "Impossible," said he. "Please to order a new one." Thereupon
Akaky Akakiyevich handed over the ten-kopek piece. "Thank you, sir. I
will drink your good health," said Petrovich. "But as for the cloak,
don't trouble yourself about it; it is good for nothing. I will make
you a capital new one, so let us settle about it now."

Akaky Akakiyevich was still for mending it, but Petrovich would not
hear of it, and said, "I shall certainly have to make you a new one,
and you may depend upon it that I shall do my best. It may even be, as
the fashion goes, that the collar can be fastened by silver hooks
under a flap."

Then Akaky Akakiyevich saw that it was impossible to get along without
a new cloak, and his spirit sank utterly. How, in fact, was it to be
done? Where was the money to come from? He must have some new
trousers, and pay a debt of long standing to the shoemaker for putting
new tops to his old boots, and he must order three shirts from the
seamstress, and a couple of pieces of linen. In short, all his money
must be spent. And even if the director should be so kind as to order
him to receive forty-five or even fifty rubles instead of forty, it
would be a mere nothing, a mere drop in the ocean towards the funds
necessary for a cloak, although he knew that Petrovich was often
wrong-headed enough to blurt out some outrageous price, so that even
his own wife could not refrain from exclaiming, "Have you lost your
senses, you fool?" At one time he would not work at any price, and now
it was quite likely that he had named a higher sum than the cloak
would cost.

But although he knew that Petrovich would undertake to make a cloak
for eighty rubles, still, where was he to get the eighty rubles from?
He might possibly manage half. Yes, half might be procured, but where
was the other half to come from? But the reader must first be told
where the first half came from.

Akaky Akakiyevich had a habit of putting, for every ruble he spent, a
groschen into a small box, fastened with lock and key, and with a slit
in the top for the reception of money. At the end of every half-year
he counted over the heap of coppers, and changed it for silver. This
he had done for a long time, and in the course of years, the sum had
mounted up to over forty rubles. Thus he had one half on hand. But
where was he to find the other half? Where was he to get another forty
rubles from? Akaky Akakiyevich thought and thought, and decided that
it would be necessary to curtail his ordinary expenses, for the space
of one year at least, to dispense with tea in the evening, to burn no
candles, and, if there was anything which he must do, to go into his
landlady's room, and work by her light. When he went into the street,
he must walk as lightly as he could, and as cautiously, upon the
stones, almost upon tiptoe, in order not to wear his heels down in too
short a time. He must give the laundress as little to wash as
possible; and, in order not to wear out his clothes, he must take them
off as soon as he got home, and wear only his cotton dressing-gown,
which had been long and carefully saved.

To tell the truth, it was a little hard for him at first to accustom
himself to these deprivations. But he got used to them at length,
after a fashion, and all went smoothly. He even got used to being
hungry in the evening, but he made up for it by treating himself, so
to say, in spirit, by bearing ever in mind the idea of his future
cloak. From that time forth, his existence seemed to become, in some
way, fuller, as if he were married, or as if some other man lived in
him, as if, in fact, he were not alone, and some pleasant friend had
consented to travel along life's path with him, the friend being no
other than the cloak, with thick wadding and a strong lining incapable
of wearing out. He became more lively, and even his character grew
firmer, like that of a man who has made up his mind, and set himself a
goal. From his face and gait, doubt and indecision, all hesitating and
wavering disappeared of themselves. Fire gleamed in his eyes, and
occasionally the boldest and most daring ideas flitted through his
mind. Why not, for instance, have marten fur on the collar? The
thought of this almost made him absent-minded. Once, in copying a
letter, he nearly made a mistake, so that he exclaimed almost aloud,
"Ugh!" and crossed himself. Once, in the course of every month, he had
a conference with Petrovich on the subject of the cloak, where it
would be better to buy the cloth, and the colour, and the price. He
always returned home satisfied, though troubled, reflecting that the
time would come at last when it could all be bought, and then the
cloak made.

The affair progressed more briskly than he had expected. For beyond
all his hopes, the director awarded neither forty nor forty-five
rubles for Akaky Akakiyevich's share, but sixty. Whether he suspected
that Akaky Akakiyevich needed a cloak, or whether it was merely
chance, at all events, twenty extra rubles were by this means
provided. This circumstance hastened matters. Two or three months more
of hunger and Akaky Akakiyevich had accumulated about eighty rubles.
His heart, generally so quiet, began to throb. On the first possible
day, he went shopping in company with Petrovich. They bought some very
good cloth, and at a reasonable rate too, for they had been
considering the matter for six months, and rarely let a month pass
without their visiting the shops to enquire prices. Petrovich himself
said that no better cloth could be had. For lining, they selected a
cotton stuff, but so firm and thick, that Petrovich declared it to be
better than silk, and even prettier and more glossy. They did not buy
the marten fur, because it was, in fact, dear, but in its stead, they
picked out the very best of cat-skin which could be found in the shop,
and which might, indeed, be taken for marten at a distance.

Petrovich worked at the cloak two whole weeks, for there was a great
deal of quilting; otherwise it would have been finished sooner. He
charged twelve rubles for the job, it could not possibly have been
done for less. It was all sewed with silk, in small, double seams, and
Petrovich went over each seam afterwards with his own teeth, stamping
in various patterns.

It was--it is difficult to say precisely on what day, but probably the
most glorious one in Akaky Akakiyevich's life, when Petrovich at
length brought home the cloak. He brought it in the morning, before
the hour when it was necessary to start for the department. Never did
a cloak arrive so exactly in the nick of time, for the severe cold had
set in, and it seemed to threaten to increase. Petrovich brought the
cloak himself as befits a good tailor. On his countenance was a
significant expression, such as Akaky Akakiyevich had never beheld
there. He seemed fully sensible that he had done no small deed, and
crossed a gulf separating tailors who put in linings, and execute
repairs, from those who make new things. He took the cloak out of the
pocket-handkerchief in which he had brought it. The handkerchief was
fresh from the laundress, and he put it in his pocket for use. Taking
out the cloak, he gazed proudly at it, held it up with both hands, and
flung it skilfully over the shoulders of Akaky Akakiyevich. Then he
pulled it and fitted it down behind with his hand, and he draped it
around Akaky Akakiyevich without buttoning it. Akaky Akakiyevich, like
an experienced man, wished to try the sleeves. Petrovich helped him on
with them, and it turned out that the sleeves were satisfactory also.
In short, the cloak appeared to be perfect, and most seasonable.
Petrovich did not neglect to observe that it was only because he lived
in a narrow street, and had no signboard, and had known Akaky
Akakiyevich so long, that he had made it so cheaply; but that if he
had been in business on the Nevsky Prospect, he would have charged
seventy-five rubles for the making alone. Akaky Akakiyevich did not
care to argue this point with Petrovich. He paid him, thanked him, and
set out at once in his new cloak for the department. Petrovich
followed him, and pausing in the street, gazed long at the cloak in
the distance, after which he went to one side expressly to run through
a crooked alley, and emerge again into the street beyond to gaze once
more upon the cloak from another point, namely, directly in front.

Meantime Akaky Akakiyevich went on in holiday mood. He was conscious
every second of the time that he had a new cloak on his shoulders, and
several times he laughed with internal satisfaction. In fact, there
were two advantages, one was its warmth, the other its beauty. He saw
nothing of the road, but suddenly found himself at the department. He
took off his cloak in the ante-room, looked it over carefully, and
confided it to the special care of the attendant. It is impossible to
say precisely how it was that every one in the department knew at once
that Akaky Akakiyevich had a new cloak, and that the "cape" no longer
existed. All rushed at the same moment into the ante-room to inspect
it. They congratulated him, and said pleasant things to him, so that
he began at first to smile, and then to grow ashamed. When all
surrounded him, and said that the new cloak must be "christened," and
that he must at least give them all a party, Akaky Akakiyevich lost
his head completely, and did not know where he stood, what to answer,
or how to get out of it. He stood blushing all over for several
minutes, trying to assure them with great simplicity that it was not a
new cloak, that it was in fact the old "cape."

At length one of the officials, assistant to the head clerk, in order
to show that he was not at all proud, and on good terms with his
inferiors, said:

"So be it, only I will give the party instead of Akaky Akakiyevich; I
invite you all to tea with me to-night. It just happens to be my
name-day too."

The officials naturally at once offered the assistant clerk their
congratulations, and accepted the invitation with pleasure. Akaky
Akakiyevich would have declined; but all declared that it was
discourteous, that it was simply a sin and a shame, and that he could
not possibly refuse. Besides, the notion became pleasant to him when
he recollected that he should thereby have a chance of wearing his new
cloak in the evening also.

That whole day was truly a most triumphant festival for Akaky
Akakiyevich. He returned home in the most happy frame of mind, took
off his cloak, and hung it carefully on the wall, admiring afresh the
cloth and the lining. Then he brought out his old, worn-out cloak, for
comparison. He looked at it, and laughed, so vast was the difference.
And long after dinner he laughed again when the condition of the
"cape" recurred to his mind. He dined cheerfully, and after dinner
wrote nothing, but took his ease for a while on the bed, until it got
dark. Then he dressed himself leisurely, put on his cloak, and stepped
out into the street.

Where the host lived, unfortunately we cannot say. Our memory begins
to fail us badly. The houses and streets in St. Petersburg have become
so mixed up in our head that it is very difficult to get anything out
of it again in proper form. This much is certain, that the official
lived in the best part of the city; and therefore it must have been
anything but near to Akaky Akakiyevich's residence. Akaky Akakiyevich
was first obliged to traverse a kind of wilderness of deserted,
dimly-lighted streets. But in proportion as he approached the
official's quarter of the city, the streets became more lively, more
populous, and more brilliantly illuminated. Pedestrians began to
appear; handsomely dressed ladies were more frequently encountered;
the men had otter skin collars to their coats; shabby sleigh-men with
their wooden, railed sledges stuck over with brass-headed nails,
became rarer; whilst on the other hand, more and more drivers in red
velvet caps, lacquered sledges and bear-skin coats began to appear,
and carriages with rich hammer-cloths flew swiftly through the
streets, their wheels scrunching the snow.

Akaky Akakiyevich gazed upon all this as upon a novel sight. He had
not been in the streets during the evening for years. He halted out of
curiosity before a shop-window, to look at a picture representing a
handsome woman, who had thrown off her shoe, thereby baring her whole
foot in a very pretty way; whilst behind her the head of a man with
whiskers and a handsome moustache peeped through the doorway of
another room. Akaky Akakiyevich shook his head, and laughed, and then
went on his way. Why did he laugh? Either because he had met with a
thing utterly unknown, but for which every one cherishes,
nevertheless, some sort of feeling, or else he thought, like many
officials, "Well, those French! What is to be said? If they do go in
for anything of that sort, why--" But possibly he did not think at
all.

Akaky Akakiyevich at length reached the house in which the head
clerk's assistant lodged. He lived in fine style. The staircase was
lit by a lamp, his apartment being on the second floor. On entering
the vestibule, Akaky Akakiyevich beheld a whole row of goloshes on the
floor. Among them, in the centre of the room, stood a samovar, humming
and emitting clouds of steam. On the walls hung all sorts of coats and
cloaks, among which there were even some with beaver collars, or
velvet facings. Beyond, the buzz of conversation was audible, and
became clear and loud, when the servant came out with a trayful of
empty glasses, cream-jugs and sugar-bowls. It was evident that the
officials had arrived long before, and had already finished their
first glass of tea.

Akaky Akakiyevich, having hung up his own cloak, entered the inner
room. Before him all at once appeared lights, officials, pipes, and
card-tables, and he was bewildered by a sound of rapid conversation
rising from all the tables, and the noise of moving chairs. He halted
very awkwardly in the middle of the room, wondering what he ought to
do. But they had seen him. They received him with a shout, and all
thronged at once into the ante-room, and there took another look at
his cloak. Akaky Akakiyevich, although somewhat confused, was
frank-hearted, and could not refrain from rejoicing when he saw how
they praised his cloak. Then, of course, they all dropped him and his
cloak, and returned, as was proper, to the tables set out for whist.

All this, the noise, the talk, and the throng of people, was rather
overwhelming to Akaky Akakiyevich. He simply did not know where he
stood, or where to put his hands, his feet, and his whole body.
Finally he sat down by the players, looked at the cards, gazed at the
face of one and another, and after a while began to gape, and to feel
that it was wearisome, the more so, as the hour was already long past
when he usually went to bed. He wanted to take leave of the host, but
they would not let him go, saying that he must not fail to drink a
glass of champagne, in honour of his new garment. In the course of an
hour, supper, consisting of vegetable salad, cold veal, pastry,
confectioner's pies, and champagne, was served. They made Akaky
Akakiyevich drink two glasses of champagne, after which he felt things
grow livelier.

Still, he could not forget that it was twelve o'clock, and that he
should have been at home long ago. In order that the host might not
think of some excuse for detaining him, he stole out of the room
quickly, sought out, in the ante-room, his cloak, which, to his
sorrow, he found lying on the floor, brushed it, picked off every
speck upon it, put it on his shoulders, and descended the stairs to
the street.

In the street all was still bright. Some petty shops, those permanent
clubs of servants and all sorts of folks, were open. Others were shut,
but, nevertheless, showed a streak of light the whole length of the
door-crack, indicating that they were not yet free of company, and
that probably some domestics, male and female, were finishing their
stories and conversations, whilst leaving their masters in complete
ignorance as to their whereabouts. Akaky Akakiyevich went on in a
happy frame of mind. He even started to run, without knowing why,
after some lady, who flew past like a flash of lightning. But he
stopped short, and went on very quietly as before, wondering why he
had quickened his pace. Soon there spread before him these deserted
streets which are not cheerful in the daytime, to say nothing of the
evening. Now they were even more dim and lonely. The lanterns began to
grow rarer, oil, evidently, had been less liberally supplied. Then
came wooden houses and fences. Not a soul anywhere; only the snow
sparkled in the streets, and mournfully veiled the low-roofed cabins
with their closed shutters. He approached the spot where the street
crossed a vast square with houses barely visible on its farther side,
a square which seemed a fearful desert.

Afar, a tiny spark glimmered from some watchman's-box, which seemed to
stand on the edge of the world. Akaky Akakiyevich's cheerfulness
diminished at this point in a marked degree. He entered the square,
not without an involuntary sensation of fear, as though his heart
warned him of some evil. He glanced back, and on both sides it was
like a sea about him. "No, it is better not to look," he thought, and
went on, closing his eyes. When he opened them, to see whether he was
near the end of the square, he suddenly beheld, standing just before
his very nose, some bearded individuals of precisely what sort, he
could not make out. All grew dark before his eyes, and his heart
throbbed.

"Of course, the cloak is mine!" said one of them in a loud voice,
seizing hold of his collar. Akaky Akakiyevich was about to shout
"Help!" when the second man thrust a fist, about the size of an
official's head, at his very mouth, muttering, "Just you dare to
scream!"

Akaky Akakiyevich felt them strip off his cloak, and give him a kick.
He fell headlong upon the snow, and felt no more.

In a few minutes he recovered consciousness, and rose to his feet, but
no one was there. He felt that it was cold in the square, and that his
cloak was gone. He began to shout, but his voice did not appear to
reach the outskirts of the square. In despair, but without ceasing to
shout, he started at a run across the square, straight towards the
watch-box, beside which stood the watchman, leaning on his halberd,
and apparently curious to know what kind of a customer was running
towards him shouting. Akaky Akakiyevich ran up to him, and began in a
sobbing voice to shout that he was asleep, and attended to nothing,
and did not see when a man was robbed. The watchman replied that he
had seen two men stop him in the middle of the square, but supposed
that they were friends of his, and that, instead of scolding vainly,
he had better go to the police on the morrow, so that they might make
a search for whoever had stolen the cloak.

Akaky Akakiyevich ran home and arrived in a state of complete
disorder, his hair which grew very thinly upon his temples and the
back of his head all tousled, his body, arms and legs, covered with
snow. The old woman, who was mistress of his lodgings, on hearing a
terrible knocking, sprang hastily from her bed, and, with only one
shoe on, ran to open the door, pressing the sleeve of her chemise to
her bosom out of modesty. But when she had opened it, she fell back on
beholding Akaky Akakiyevich in such a condition. When he told her
about the affair, she clasped her hands, and said that he must go
straight to the district chief of police, for his subordinate would
turn up his nose, promise well, and drop the matter there. The very
best thing to do, therefore, would be to go to the district chief,
whom she knew, because Finnish Anna, her former cook, was now nurse at
his house. She often saw him passing the house, and he was at church
every Sunday, praying, but at the same time gazing cheerfully at
everybody; so that he must be a good man, judging from all
appearances. Having listened to this opinion, Akaky Akakiyevich betook
himself sadly to his room. And how he spent the night there, any one
who can put himself in another's place may readily imagine.

Early in the morning, he presented himself at the district chief's,
but was told the official was asleep. He went again at ten and was
again informed that he was asleep. At eleven, and they said, "The
superintendent is not at home." At dinner time, and the clerks in the
ante-room would not admit him on any terms, and insisted upon knowing
his business. So that at last, for once in his life, Akaky Akakiyevich
felt an inclination to show some spirit, and said curtly that he must
see the chief in person, that they ought not to presume to refuse him
entrance, that he came from the department of justice, and that when
he complained of them, they would see.

The clerks dared make no reply to this, and one of them went to call
the chief, who listened to the strange story of the theft of the coat.
Instead of directing his attention to the principal points of the
matter, he began to question Akaky Akakiyevich. Why was he going home
so late? Was he in the habit of doing so, or had he been to some
disorderly house? So that Akaky Akakiyevich got thoroughly confused,
and left him, without knowing whether the affair of his cloak was in
proper train or not.

All that day, for the first time in his life, he never went near the
department. The next day he made his appearance, very pale, and in his
old cape, which had become even more shabby. The news of the robbery
of the cloak touched many, although there were some officials present
who never lost an opportunity, even such a one as the present, of
ridiculing Akaky Akakiyevich. They decided to make a collection for
him on the spot, but the officials had already spent a great deal in
subscribing for the director's portrait, and for some book, at the
suggestion of the head of that division, who was a friend of the
author; and so the sum was trifling.

One of them, moved by pity, resolved to help Akaky Akakiyevich with
some good advice, at least, and told him that he ought not to go to
the police, for although it might happen that a police-officer,
wishing to win the approval of his superiors, might hunt up the cloak
by some means, still, his cloak would remain in the possession of the
police if he did not offer legal proof that it belonged to him. The
best thing for him, therefore, would be to apply to a certain
prominent personage; since this prominent personage, by entering into
relation with the proper persons, could greatly expedite the matter.

As there was nothing else to be done, Akaky Akakiyevich decided to go
to the prominent personage. What was the exact official position of
the prominent personage, remains unknown to this day. The reader must
know that the prominent personage had but recently become a prominent
personage, having up to that time been only an insignificant person.
Moreover, his present position was not considered prominent in
comparison with others still more so. But there is always a circle of
people to whom what is insignificant in the eyes of others, is
important enough. Moreover, he strove to increase his importance by
sundry devices. For instance, he managed to have the inferior
officials meet him on the staircase when he entered upon his service;
no one was to presume to come directly to him, but the strictest
etiquette must be observed; the collegiate recorder must make a report
to the government secretary, the government secretary to the titular
councillor, or whatever other man was proper, and all business must
come before him in this manner. In Holy Russia, all is thus
contaminated with the love of imitation; every man imitates and copies
his superior. They even say that a certain titular councillor, when
promoted to the head of some small separate office, immediately
partitioned off a private room for himself, called it the audience
chamber, and posted at the door a lackey with red collar and braid,
who grasped the handle of the door, and opened to all comers, though
the audience chamber would hardly hold an ordinary writing table.

The manners and customs of the prominent personage were grand and
imposing, but rather exaggerated. The main foundation of his system
was strictness. "Strictness, strictness, and always strictness!" he
generally said; and at the last word he looked significantly into the
face of the person to whom he spoke. But there was no necessity for
this, for the halfscore of subordinates, who formed the entire force
of the office, were properly afraid. On catching sight of him afar
off, they left their work, and waited, drawn up in line, until he had
passed through the room. His ordinary converse with his inferiors
smacked of sternness, and consisted chiefly of three phrases: "How
dare you?" "Do you know whom you are speaking to?" "Do you realise who
is standing before you?"

Otherwise he was a very kind-hearted man, good to his comrades, and
ready to oblige. But the rank of general threw him completely off his
balance. On receiving any one of that rank, he became confused, lost
his way, as it were, and never knew what to do. If he chanced to be
amongst his equals, he was still a very nice kind of man, a very good
fellow in many respects, and not stupid, but the very moment that he
found himself in the society of people but one rank lower than
himself, he became silent. And his situation aroused sympathy, the
more so, as he felt himself that he might have been making an
incomparably better use of his time. In his eyes, there was sometimes
visible a desire to join some interesting conversation or group, but
he was kept back by the thought, "Would it not be a very great
condescension on his part? Would it not be familiar? And would he not
thereby lose his importance?" And in consequence of such reflections,
he always remained in the same dumb state, uttering from time to time
a few monosyllabic sounds, and thereby earning the name of the most
wearisome of men.

To this prominent personage Akaky Akakiyevich presented himself, and
this at the most unfavourable time for himself, though opportune for
the prominent personage. The prominent personage was in his cabinet,
conversing very gaily with an old acquaintance and companion of his
childhood, whom he had not seen for several years, and who had just
arrived, when it was announced to him that a person named Bashmachkin
had come. He asked abruptly, "Who is he?"--"Some official," he was
informed. "Ah, he can wait! This is no time for him to call," said the
important man.

It must be remarked here that the important man lied outrageously. He
had said all he had to say to his friend long before, and the
conversation had been interspersed for some time with very long
pauses, during which they merely slapped each other on the leg, and
said, "You think so, Ivan Abramovich!" "Just so, Stepan Varlamovich!"
Nevertheless, he ordered that the official should be kept waiting, in
order to show his friend, a man who had not been in the service for a
long time, but had lived at home in the country, how long officials
had to wait in his ante-room.

At length, having talked himself completely out, and more than that,
having had his fill of pauses, and smoked a cigar in a very
comfortable arm-chair with reclining back, he suddenly seemed to
recollect, and said to the secretary, who stood by the door with
papers of reports, "So it seems that there is an official waiting to
see me. Tell him that he may come in." On perceiving Akaky
Akakiyevich's modest mien and his worn uniform, he turned abruptly to
him, and said, "What do you want?" in a curt hard voice, which he had
practised in his room in private, and before the looking-glass, for a
whole week before being raised to his present rank.

Akaky Akakiyevich, who was already imbued with a due amount of fear,
became somewhat confused, and as well as his tongue would permit,
explained, with a rather more frequent addition than usual of the word
"that" that his cloak was quite new, and had been stolen in the most
inhuman manner; that he had applied to him, in order that he might, in
some way, by his intermediation--that he might enter into
correspondence with the chief of police, and find the cloak.

For some inexplicable reason, this conduct seemed familiar to the
prominent personage.

"What, my dear sir!" he said abruptly, "are you not acquainted with
etiquette? To whom have you come? Don't you know how such matters are
managed? You should first have presented a petition to the office. It
would have gone to the head of the department, then to the chief of
the division, then it would have been handed over to the secretary,
and the secretary would have given it to me."

"But, your excellency," said Akaky Akakiyevich, trying to collect his
small handful of wits, and conscious at the same time that he was
perspiring terribly, "I, your excellency, presumed to trouble you
because secretaries--are an untrustworthy race."

"What, what, what!" said the important personage. "Where did you get
such courage? Where did you get such ideas? What impudence towards
their chiefs and superiors has spread among the young generation!" The
prominent personage apparently had not observed that Akaky Akakiyevich
was already in the neighbourhood of fifty. If he could be called a
young man, it must have been in comparison with some one who was
seventy. "Do you know to whom you are speaking? Do you realise who is
standing before you? Do you realise it? Do you realise it, I ask you!"
Then he stamped his foot, and raised his voice to such a pitch that it
would have frightened even a different man from Akaky Akakiyevich.

Akaky Akakiyevich's senses failed him. He staggered, trembled in every
limb, and, if the porters had not run in to support him, would have
fallen to the floor. They carried him out insensible. But the
prominent personage, gratified that the effect should have surpassed
his expectations, and quite intoxicated with the thought that his word
could even deprive a man of his senses, glanced sideways at his friend
in order to see how he looked upon this, and perceived, not without
satisfaction, that his friend was in a most uneasy frame of mind, and
even beginning on his part, to feel a trifle frightened.

Akaky Akakiyevich could not remember how he descended the stairs, and
got into the street. He felt neither his hands nor feet. Never in his
life had he been so rated by any high official, let alone a strange
one. He went staggering on through the snow-storm, which was blowing
in the streets, with his mouth wide open. The wind, in St. Petersburg
fashion, darted upon him from all quarters, and down every
cross-street. In a twinkling it had blown a quinsy into his throat,
and he reached home unable to utter a word. His throat was swollen,
and he lay down on his bed. So powerful is sometimes a good scolding!

The next day a violent fever developed. Thanks to the generous
assistance of the St. Petersburg climate, the malady progressed more
rapidly than could have been expected, and when the doctor arrived, he
found, on feeling the sick man's pulse, that there was nothing to be
done, except to prescribe a poultice, so that the patient might not be
left entirely without the beneficent aid of medicine. But at the same
time, he predicted his end in thirty-six hours. After this he turned
to the landlady, and said, "And as for you, don't waste your time on
him. Order his pine coffin now, for an oak one will be too expensive
for him."

Did Akaky Akakiyevich hear these fatal words? And if he heard them,
did they produce any overwhelming effect upon him? Did he lament the
bitterness of his life?--We know not, for he continued in a delirious
condition. Visions incessantly appeared to him, each stranger than the
other. Now he saw Petrovich, and ordered him to make a cloak, with
some traps for robbers, who seemed to him to be always under the bed;
and he cried every moment to the landlady to pull one of them from
under his coverlet. Then he inquired why his old mantle hung before
him when he had a new cloak. Next he fancied that he was standing
before the prominent person, listening to a thorough setting-down and
saying, "Forgive me, your excellency!" but at last he began to curse,
uttering the most horrible words, so that his aged landlady crossed
herself, never in her life having heard anything of the kind from him,
and more so as these words followed directly after the words "your
excellency." Later on he talked utter nonsense, of which nothing could
be made, all that was evident being that these incoherent words and
thoughts hovered ever about one thing, his cloak.

At length poor Akaky Akakiyevich breathed his last. They sealed up
neither his room nor his effects, because, in the first place, there
were no heirs, and, in the second, there was very little to inherit
beyond a bundle of goose-quills, a quire of white official paper,
three pairs of socks, two or three buttons which had burst off his
trousers, and the mantle already known to the reader. To whom all this
fell, God knows. I confess that the person who told me this tale took
no interest in the matter. They carried Akaky Akakiyevich out, and
buried him.

And St. Petersburg was left without Akaky Akakiyevich, as though he
had never lived there. A being disappeared, who was protected by none,
dear to none, interesting to none, and who never even attracted to
himself the attention of those students of human nature who omit no
opportunity of thrusting a pin through a common fly and examining it
under the microscope. A being who bore meekly the jibes of the
department, and went to his grave without having done one unusual
deed, but to whom, nevertheless, at the close of his life, appeared a
bright visitant in the form of a cloak, which momentarily cheered his
poor life, and upon him, thereafter, an intolerable misfortune
descended, just as it descends upon the heads of the mighty of this
world!

Several days after his death, the porter was sent from the department
to his lodgings, with an order for him to present himself there
immediately, the chief commanding it. But the porter had to return
unsuccessful, with the answer that he could not come; and to the
question, "Why?" replied, "Well, because he is dead! he was buried
four days ago." In this manner did they hear of Akaky Akakiyevich's
death at the department. And the next day a new official sat in his
place, with a handwriting by no means so upright, but more inclined
and slanting.

But who could have imagined that this was not really the end of Akaky
Akakiyevich, that he was destined to raise a commotion after death, as
if in compensation for his utterly insignificant life? But so it
happened, and our poor story unexpectedly gains a fantastic ending.

A rumour suddenly spread through St. Petersburg, that a dead man had
taken to appearing on the Kalinkin Bridge, and its vicinity, at night
in the form of an official seeking a stolen cloak, and that, under the
pretext of its being the stolen cloak, he dragged, without regard to
rank or calling, every one's cloak from his shoulders, be it cat-skin,
beaver, fox, bear, sable, in a word, every sort of fur and skin which
men adopted for their covering. One of the department officials saw
the dead man with his own eyes, and immediately recognised in him
Akaky Akakiyevich. This, however, inspired him with such terror, that
he ran off with all his might, and therefore did not scan the dead man
closely, but only saw how the latter threatened him from afar with his
finger. Constant complaints poured in from all quarters, that the
backs and shoulders, not only of titular but even of court
councillors, were exposed to the danger of a cold, on account of the
frequent dragging off of their cloaks.

Arrangements were made by the police to catch the corpse, alive or
dead, at any cost, and punish him as an example to others, in the most
severe manner. In this they nearly succeeded, for a watchman, on guard
in Kirinshkin Lane, caught the corpse by the collar on the very scene
of his evil deeds, when attempting to pull off the frieze cloak of a
retired musician. Having seized him by the collar, he summoned, with a
shout, two of his comrades, whom he enjoined to hold him fast, while
he himself felt for a moment in his boot, in order to draw out his
snuff-box, and refresh his frozen nose. But the snuff was of a sort
which even a corpse could not endure. The watchman having closed his
right nostril with his finger, had no sooner succeeded in holding half
a handful up to the left, than the corpse sneezed so violently that he
completely filled the eyes of all three. While they raised their hands
to wipe them, the dead man vanished completely, so that they
positively did not know whether they had actually had him in their
grip at all. Thereafter the watchmen conceived such a terror of dead
men that they were afraid even to seize the living, and only screamed
from a distance. "Hey, there! go your way!" So the dead official began
to appear even beyond the Kalinkin Bridge, causing no little terror to
all timid people.

But we have totally neglected that certain prominent personage who may
really be considered as the cause of the fantastic turn taken by this
true history. First of all, justice compels us to say, that after the
departure of poor, annihilated Akaky Akakiyevich, he felt something
like remorse. Suffering was unpleasant to him, for his heart was
accessible to many good impulses, in spite of the fact that his rank
often prevented his showing his true self. As soon as his friend had
left his cabinet, he began to think about poor Akaky Akakiyevich. And
from that day forth, poor Akaky Akakiyevich, who could not bear up
under an official reprimand, recurred to his mind almost every day.
The thought troubled him to such an extent, that a week later he even
resolved to send an official to him, to learn whether he really could
assist him. And when it was reported to him that Akaky Akakiyevich had
died suddenly of fever, he was startled, hearkened to the reproaches
of his conscience, and was out of sorts for the whole day.

Wishing to divert his mind in some way and drive away the disagreeable
impression, he set out that evening for one of his friends' houses,
where he found quite a large party assembled. What was better, nearly
every one was of the same rank as himself, so that he need not feel in
the least constrained. This had a marvellous effect upon his mental
state. He grew expansive, made himself agreeable in conversation, in
short, he passed a delightful evening. After supper he drank a couple
of glasses of champagne--not a bad recipe for cheerfulness, as every
one knows. The champagne inclined him to various adventures, and he
determined not to return home, but to go and see a certain well-known
lady, of German extraction, Karolina Ivanovna, a lady, it appears,
with whom he was on a very friendly footing.

It must be mentioned that the prominent personage was no longer a
young man, but a good husband and respected father of a family. Two
sons, one of whom was already in the service, and a good-looking,
sixteen-year-old daughter, with a slightly arched but pretty little
nose, came every morning to kiss his hand and say, "\emph{Bon jour}, papa."
His wife, a still fresh and good-looking woman, first gave him her
hand to kiss, and then, reversing the procedure, kissed his. But the
prominent personage, though perfectly satisfied in his domestic
relations, considered it stylish to have a friend in another quarter
of the city. This friend was scarcely prettier or younger than his
wife; but there are such puzzles in the world, and it is not our place
to judge them. So the important personage descended the stairs,
stepped into his sledge, said to the coachman, "To Karolina
Ivanovna's," and, wrapping himself luxuriously in his warm cloak,
found himself in that delightful frame of mind than which a Russian
can conceive nothing better, namely, when you think of nothing
yourself, yet when the thoughts creep into your mind of their own
accord, each more agreeable than the other, giving you no trouble
either to drive them away, or seek them. Fully satisfied, he recalled
all the gay features of the evening just passed and all the mots which
had made the little circle laugh. Many of them he repeated in a low
voice, and found them quite as funny as before; so it is not
surprising that he should laugh heartily at them. Occasionally,
however, he was interrupted by gusts of wind, which, coming suddenly,
God knows whence or why, cut his face, drove masses of snow into it,
filled out his cloak-collar like a sail, or suddenly blew it over his
head with supernatural force, and thus caused him constant trouble to
disentangle himself.

Suddenly the important personage felt some one clutch him firmly by
the collar. Turning round, he perceived a man of short stature, in an
old, worn uniform, and recognised, not without terror, Akaky
Akakiyevich. The official's face was white as snow, and looked just
like a corpse's. But the horror of the important personage transcended
all bounds when he saw the dead man's mouth open, and heard it utter
the following remarks, while it breathed upon him the terrible odour
of the grave: "Ah, here you are at last! I have you, that--by the
collar! I need your cloak. You took no trouble about mine, but
reprimanded me. So now give up your own."

The pallid prominent personage almost died of fright. Brave as he was
in the office and in the presence of inferiors generally, and
although, at the sight of his manly form and appearance, every one
said, "Ugh! how much character he has!" at this crisis, he, like many
possessed of an heroic exterior, experienced such terror, that, not
without cause, he began to fear an attack of illness. He flung his
cloak hastily from his shoulders and shouted to his coachman in an
unnatural voice, "Home at full speed!" The coachman, hearing the tone
which is generally employed at critical moments, and even accompanied
by something much more tangible, drew his head down between his
shoulders in case of an emergency, flourished his whip, and flew on
like an arrow. In a little more than six minutes the prominent
personage was at the entrance of his own house. Pale, thoroughly
scared, and cloakless, he went home instead of to Karolina Ivanovna's,
reached his room somehow or other, and passed the night in the direst
distress; so that the next morning over their tea, his daughter said,
"You are very pale to-day, papa." But papa remained silent, and said
not a word to any one of what had happened to him, where he had been,
or where he had intended to go.

This occurrence made a deep impression upon him. He even began to say,
"How dare you? Do you realise who is standing before you?" less
frequently to the under-officials, and, if he did utter the words, it
was only after first having learned the bearings of the matter. But
the most noteworthy point was, that from that day forward the
apparition of the dead official ceased to be seen. Evidently the
prominent personage's cloak just fitted his shoulders. At all events,
no more instances of his dragging cloaks from people's shoulders were
heard of. But many active and solicitous persons could by no means
reassure themselves, and asserted that the dead official still showed
himself in distant parts of the city.

In fact, one watchman in Kolomen saw with his own eyes the apparition
come from behind a house. But the watchman was not a strong man, so he
was afraid to arrest him, and followed him in the dark, until, at
length, the apparition looked round, paused, and inquired, "What do
you want?" at the same time showing such a fist as is never seen on
living men. The watchman said, "Nothing," and turned back instantly.
But the apparition was much too tall, wore huge moustaches, and,
directing its steps apparently towards the Obukhov Bridge, disappeared
in the darkness of the night.

