\chapter{The Shades, A Phantasy\\
\small \hspace{20pt}
by Vladimir G. Korlenko}

\section{I}

\lettrine[lines=3,lhang=0.11,lraise=0,loversize=0.05]{A}{}%
month and two days had elapsed since the judges, amid the loud
acclaim of the Athenian people, had pronounced the death sentence
against the philosopher Socrates because he had sought to destroy
faith in the gods. What the gadfly is to the horse Socrates was to
Athens. The gadfly stings the horse in order to prevent it from dozing
off and to keep it moving briskly on its course. The philosopher said
to the people of Athens:

"I am your gadfly. My sting pricks your conscience and arouses you
when you are caught napping. Sleep not, sleep not, people of Athens;
awake and seek the truth!"

The people arose in their exasperation and cruelly demanded to be rid
of their gadfly.

"Perchance both of his accusers, Meletus and Anytus, are wrong," said
the citizens, on leaving the court after sentence had been pronounced.

"But after all whither do his doctrines tend? What would he do? He has
wrought confusion, he overthrows beliefs that have existed since the
beginning, he speaks of new virtues which must be recognised and
sought for, he speaks of a Divinity hitherto unknown to us. The
blasphemer, he deems himself wiser than the gods! No, 'twere better we
remain true to the old gods whom we know. They may not always be just,
sometimes they may flare up in unjust wrath, and they may also be
seized with a wanton lust for the wives of mortals; but did not our
ancestors live with them in the peace of their souls, did not our
forefathers accomplish their heroic deeds with the help of these very
gods? And now the faces of the Olympians have paled and the old virtue
is out of joint. What does it all lead to? Should not an end be put to
this impious wisdom once for all?"

Thus the citizens of Athens spoke to one another as they left the
place, and the blue twilight was falling. They had determined to kill
the restless gadfly in the hope that the countenances of the gods
would shine again. And yet--before their souls arose the mild figure
of the singular philosopher. There were some citizens who recalled how
courageously he had shared their troubles and dangers at Potidæa; how
he alone had prevented them from committing the sin of unjustly
executing the generals after the victory over the Arginusæe; how he
alone had dared to raise his voice against the tyrants who had had
fifteen hundred people put to death, speaking to the people on the
market-place concerning shepherds and their sheep.

"Is not he a good shepherd," he asked, "who guards his flock and
watches over its increase? Or is it the work of the good shepherd to
reduce the number of his sheep and disperse them, and of the good
ruler to do the same with his people? Men of Athens, let us
investigate this question!"

And at this question of the solitary, undefended philosopher, the
faces of the tyrants paled, while the eyes of the youths kindled with
the fire of just wrath and indignation.

Thus, when on dispersing after the sentence the Athenians recalled all
these things of Socrates, their hearts were oppressed with heavy
doubt.

"Have we not done a cruel wrong to the son of Sophroniscus?"

But then the good Athenians looked upon the harbour and the sea, and
in the red glow of the dying day they saw the purple sails of the
sharp-keeled ship, sent to the Delian festival, shimmering in the
distance on the blue Pontus. The ship would not return until the
expiration of a month, and the Athenians recollected that during this
time no blood might be shed in Athens, whether the blood of the
innocent or the guilty. A month, moreover, has many days and still
more hours. Supposing the son of Sophroniscus had been unjustly
condemned, who would hinder his escaping from the prison, especially
since he had numerous friends to help him? Was it so difficult for the
rich Plato, for Æschines and others to bribe the guards? Then the
restless gadfly would flee from Athens to the barbarians in Thessaly,
or to the Peloponnesus, or, still farther, to Egypt; Athens would no
longer hear his blasphemous speeches; his death would not weigh upon
the conscience of the worthy citizens, and so everything would end for
the best of all.

Thus said many to themselves that evening, while aloud they praised
the wisdom of the demos and the heliasts. In secret, however, they
cherished the hope that the restless philosopher would leave Athens,
fly from the hemlock to the barbarians, and so free the Athenians of
his troublesome presence and of the pangs of consciences that smote
them for inflicting death upon an innocent man.

Two and thirty times since that evening had the sun risen from the
ocean and dipped down into it again. The ship had returned from Delos
and lay in the harbour with sadly drooping sails, as if ashamed of its
native city. The moon did not shine in the heavens, the sea heaved
under a heavy fog, and on the hills lights peered through the
obscurity like the eyes of men gripped by a sense of guilt.

The stubborn Socrates did not spare the conscience of the good
Athenians.

"We part! You go home and I go to death," he said to the judges after
the sentence had been pronounced. "I know not, my friends, which of us
chooses the better lot!"

As the time had approached for the return of the ship, many of the
citizens had begun to feel uneasy. Must that obstinate fellow really
die? And they began to appeal to the consciences of Æschines, Phædo,
and other pupils of Socrates, trying to urge them on to further
efforts for their master.

"Will you permit your teacher to die?" they asked reproachfully in
biting tones. "Or do you grudge the few coins it would take to bribe
the guard?"

In vain Crito besought Socrates to take to flight, and complained that
the public, was upbraiding his disciples with lack of friendship and
with avarice. The self-willed philosopher refused to gratify his
pupils or the good people of Athens.

"Let us investigate." he said. "If it turns out that I must flee, I
will flee; but if I must die, I will die. Let us remember what we once
said--the wise man need not fear death, he need fear nothing but
falsehood. Is it right to abide by the laws we ourselves have made so
long as they are agreeable to us, and refuse to obey those which are
disagreeable? If my memory does not deceive me I believe we once spoke
of these things, did we not?"

"Yes, we did," answered his pupil.

"And I think all were agreed as to the answer?"

"Yes."

"But perhaps what is true for others is not true for us?"

"No, truth is alike for all, including ourselves."

"But perhaps when \emph{we} must die and not some one else, truth becomes
untruth?"

"No, Socrates, truth remains the truth under all circumstances."

After his pupil had thus agreed to each premise of Socrates in turn,
he smiled and drew his conclusion.

"If that is so, my friend, mustn't I die? Or has my head already
become so weak that I am no longer in a condition to draw a logical
conclusion? Then correct me, my friend and show my erring brain the
right way."

His pupil covered his face with his mantle and turned aside.

"Yes," he said, "now I see you must die."

And on that evening when the sea tossed hither and thither and roared
dully under the load of fog, and the whimsical wind in mournful
astonishment gently stirred the sails of the ships; when the citizens
meeting on the streets asked one another: "Is he dead?" and their
voices timidly betrayed the hope that he was not dead; when the first
breath of awakened conscience, touched the hearts of the Athenians
like the first messenger of the storm; and when, it seemed the very
faces of the gods were darkened with shame--on that evening at the
sinking of the sun the self-willed man drank the cup of death!

The wind increased in violence and shrouded the city more closely in
the veil of mist, angrily tugging at the sails of the vessels delayed
in the harbour. And the Erinyes sang their gloomy songs to the hearts
of the citizens and whipped up in their breasts that tempest which was
later, to overwhelm the denouncers of Socrates.

But in that hour the first stirrings of regret were still uncertain
and confused. The citizens found more fault with Socrates than ever
because he had not given them the satisfaction of fleeing to Thessaly;
they were annoyed with his pupils because in the last days they had
walked about in sombre mourning attire, a living reproach to the
Athenians; they were vexed with the judges because they had not had
the sense and the courage to resist the blind rage of the excited
people; they bore even the gods resentment.

"To you, ye gods, have we brought this sacrifice," spoke many.
"Rejoice, ye unsatiable!"

"I know not which of us chooses the better lot!"

Those words of Socrates came back to their memory, those his last
words to the judges and to the people gathered in the court. Now he
lay in the prison quiet and motionless under his cloak, while over the
city hovered mourning, horror, and shame.

Again he became the tormentor of the city, he who was himself no
longer accessible to torment. The gadfly had been killed, but it stung
the people more sharply than ever--sleep not, sleep not this night, O
men of Athens! Sleep not! You have committed an injustice, a cruel
injustice, which can never be erased!



\section{II}


\lettrine[lines=3,lhang=0.11,lraise=0,loversize=0.05]{D}{}%
uring those sad days Xenophon, the general, a pupil of Socrates, was
marching with his Ten Thousand in a distant land, amid dangers,
seeking a way of return to his beloved fatherland.

Æschines, Crito, Critobulus, Phædo, and Apollodorus were now occupied
with the preparations for the modest funeral.

Plato was burning his lamp and bending over a parchment; the best
disciple of the philosopher was busy inscribing the deeds, words, and
teachings that marked the end of the sage's life. A thought is never
lost, and the truth discovered by a great intellect illumines the way
for future generations like a torch in the dark.

There was one other disciple of Socrates. Not long before, the
impetuous Ctesippus had been one of the most frivolous and
pleasure-seeking of the Athenian youths. He had set up beauty as his
sole god, and had bowed before Clinias as its highest exemplar. But
since he had become acquainted with Socrates, all desire for pleasure
and all light-mindedness had gone from him. He looked on indifferently
while others took his place with Clinias. The grace of thought and the
harmony of spirit that he found in Socrates seemed a hundred times
more attractive than the graceful form and the harmonious features of
Clinias. With all the intensity of his stormy temperament he hung on
the man who had disturbed the serenity of his virginal soul, which for
the first time opened to doubts as the bud of a young oak opens to the
fresh winds of spring.

Now that the master was dead, he could find peace neither at his own
hearth nor in the oppressive stillness of the streets nor among his
friends and fellow-disciples. The gods of hearth and home and the gods
of the people inspired him with repugnance.

"I know not," he said, "whether ye are the best of all the gods to
whom numerous generations have burned incense and brought offerings;
all I know is that for your sake the blind mob extinguished the clear
torch of truth, and for your sake sacrificed the greatest and best of
mortals!"

It almost seemed to Ctesippus as though the streets and market-places
still echoed with the shrieking of that unjust sentence. And he
remembered how it was here that the people clamoured for the execution
of the generals who had led them to victory against the Argunisæ, and
how Socrates alone had opposed the savage sentence of the judges and
the blind rage of the mob. But when Socrates himself needed a
champion, no one had been found to defend him with equal strength.
Ctesippus blamed himself and his friends, and for that reason he
wanted to avoid everybody--even himself, if possible.

That evening he went to the sea. But his grief grew only the more
violent. It seemed to him that the mourning daughters of Nereus were
tossing hither and thither on the shore bewailing the death of the
best of the Athenians and the folly of the frenzied city. The waves
broke on the rocky coast with a growl of lament. Their booming sounded
like a funeral dirge.

He turned away, left the shore, and went on further without looking
before him. He forgot time and space and his own ego, filled only with
the afflicting thought of Socrates!

"Yesterday he still was, yesterday his mild words still could be
heard. How is it possible that to-day he no longer is? O night, O
giant mountain shrouded in mist, O heaving sea moved by your own life,
O restless winds that carry the breath of an immeasurable world on
your wings, O starry vault flecked with flying clouds--take me to you,
disclose to me the mystery of this death, if it is revealed to you!
And if ye know not, then grant my ignorant soul your own lofty
indifference. Remove from me these torturing questions. I no longer
have strength to carry them in my bosom without an answer, without
even the hope of an answer. For who shall answer them, now that the
lips of Socrates are sealed in eternal silence, and eternal darkness
is laid upon his lids?"

Thus Ctesippus cried out to the sea and the mountains, and to the dark
night, which followed its invariable course, ceaselessly, invisibly,
over the slumbering world. Many hours passed before Ctesippus glanced
up and saw whither his steps had unconsciously led him. A dark horror
seized his soul as he looked about him.



\section{III}


\lettrine[lines=3,lhang=0.11,lraise=0,loversize=0.05]{I}{}%
t seemed as if the unknown gods of eternal night had heard his
impious prayer. Ctesippus looked about, without being able to
recognise the place where he was. The lights of the city had long been
extinguished by the darkness. The roaring of the sea had died away in
the distance; his anxious soul had even lost the recollection of
having heard it. No single sound--no mournful cry of nocturnal bird,
nor whirr of wings, nor rustling of trees, nor murmur of a merry
stream--broke the deep silence. Only the blind will-o'-the-wisps
flickered here and there over rocks, and sheet-lightning,
unaccompanied by any sound, flared up and died down against
crag-peaks. This brief illumination merely emphasised the darkness;
and the dead light disclosed the outlines of dead deserts crossed by
gorges like crawling serpents, and rising into rocky heights in a wild
chaos.

All the joyous gods that haunt green groves, purling brooks, and
mountain valleys seemed to have fled forever from these deserts. Pan
alone, the great and mysterious Pan, was hiding somewhere nearby in
the chaos of nature, and with mocking glance seemed to be pursuing the
tiny ant that a short time before had blasphemously asked to know the
secret of the world and of death. Dark, senseless horror overwhelmed
the soul of Ctesippus. It is thus that the sea in stormy floodtide
overwhelms a rock on the shore.

Was it a dream, was it reality, or was it the revelation of the
unknown divinity? Ctesippus felt that in an instant he would step
across the threshold of life, and that his soul would melt into an
ocean of unending, inconceivable horror like a drop of rain in the
waves of the grey sea on a dark and stormy night. But at this moment
he suddenly heard voices that seemed familiar to him, and in the glare
of the sheet-lightning his eyes recognised human figures.



\section{IV}


\lettrine[lines=3,lhang=0.11,lraise=0,loversize=0.05]{O}{}%
n a rocky slope sat a man in deep despair. He had thrown a cloak over
his head and was bowed to the ground. Another figure approached him
softly, cautiously climbing upward and carefully feeling every step.
The first man uncovered his face and exclaimed:

"Is that you I just now saw, my good Socrates? Is that you passing by
me in this cheerless place? I have already spent many hours here
without knowing when day will relieve the night. I have been waiting
in vain for the dawn."

"Yes, I am Socrates, my friend, and you, are you not Elpidias who died
three days before me?"

"Yes, I am Elpidias, formerly the richest tanner in Athens, now the
most miserable of slaves. For the first time I understand the words of
the poet: 'Better to be a slave in this world than a ruler in gloomy
Hades.'"

"My friend, if it is disagreeable for you where you are, why don't you
move to another spot?"

"O Socrates, I marvel at you--how dare you wander about in this
cheerless gloom? I--I sit here overcome with grief and bemoan the joys
of a fleeting life."

"Friend Elpidias, like you, I, too, was plunged in this gloom when the
light of earthly life was removed from my eyes. But an inner voice
told me: 'Tread this new path without hesitation', and I went."

"But whither do you go, O son of Sophroniscus? Here there is no way,
no path, not even a ray of light; nothing but a chaos of rocks, mist,
and gloom."

"True. But, my Elpidias, since you are aware of this sad truth, have
you not asked yourself what is the most distressing thing in your
present situation?"

"Undoubtedly the dismal darkness."

"Then one should seek for light. Perchance you will find here the
great law--that mortals must in darkness seek the source of life. Do
you not think it is better so to seek than to remain sitting in one
spot? \emph{I} think it is, therefore I keep walking. Farewell!"

"Oh, good Socrates, abandon me not! You go with sure steps through the
pathless chaos in Hades. Hold out to me but a fold of your mantle--"

"If you think it is better for you, too, then follow me, friend
Elpidias."

And the two shades walked on, while the soul of Ctesippus, released by
sleep from its mortal envelop, flew after them, greedily absorbing the
tones of the clear Socratic speech.

"Are you here, good Socrates?" the voice of the Athenian again was
heard. "Why are you silent? Converse shortens the way, and I swear, by
Hercules, never did I have to traverse such a horrid way."

"Put questions, friend Elpidias! The question of one who seeks
knowledge brings forth answers and produces conversation."

Elpidias maintained silence for a moment, and then, after he had
collected his thoughts, asked:

"Yes, this is what I wanted to say--tell me, my poor Socrates, did
they at least give you a good burial?"

"I must confess, friend Elpidias, I cannot satisfy your curiosity."

"I understand, my poor Socrates, it doesn't help you cut a figure. Now
with me it was so different! Oh, how they buried me, how magnificently
they buried me, my poor fellow-Wanderer! I still think with great
pleasure of those lovely moments after my death. First they washed me
and sprinkled me with well-smelling balsam. Then my faithful Larissa
dressed me in garments of the finest weave. The best mourning-women of
the city tore their hair from their heads because they had been
promised good pay, and in the family vault they placed an amphora--a
crater with beautiful, decorated handles of bronze, and, besides, a
vial.--"

"Stay, friend Elpidias. I am convinced that the faithful Larissa
converted her love into several minas. Yet--"

"Exactly ten minas and four drachmas, not counting the drinks for the
guests. I hardly think that the richest tanner can come before the
souls of his ancestors and boast of such respect on the part of the
living."

"Friend Elpidias, don't you think that money would have been of more
use to the poor people who are still alive in Athens than to you at
this moment?"

"Admit, Socrates, you are speaking in envy," responded Elpidias,
pained. "I am sorry for you, unfortunate Socrates, although, between
ourselves, you really deserved your fate. I myself in the family
circle said more than once that an end ought to be put to your impious
doings, because--"

"Stay, friend, I thought you wanted to draw a conclusion, and I fear
you are straying from the straight path. Tell me, my good friend,
whither does your wavering thought tend?"

"I wanted to say that in my goodness I am sorry for you. A month ago I
myself spoke against you in the assembly, but truly none of us who
shouted so loud wanted such a great ill to befall you. Believe me, now
I am all the sorrier for you, unhappy philosopher!"

"I thank you. But tell me, my friend, do you perceive a brightness
before your eyes?"

"No, on the contrary such darkness lies before me that I must ask
myself whether this is not the misty region of Orcus."

"This way, therefore, is just as dark for you as for me?"

"Quite right."

"If I am not mistaken, you are even holding on to the folds of my
cloak?"

"Also true."

"Then we are in the same position? You see your ancestors are not
hastening to rejoice in the tale of your pompous burial. Where is the
difference between us, my good friend?"

"But, Socrates, have the gods enveloped your reason in such obscurity
that the difference is not clear to you?"

"Friend, if your situation is clearer to you, then give me your hand
and lead me, for I swear, by the dog, you let me go ahead in this
darkness."

"Cease your scoffing, Socrates! Do not make sport, and do not compare
yourself, your godless self, with a man who died in his own bed---".

"Ah, I believe I am beginning to understand you. But tell me,
Elpidias, do you hope ever again to rejoice in your bed?"

"Oh, I think not."

"And was there ever a time when you did not sleep in it?"

"Yes. That was before I bought goods from Agesilaus at half their
value. You see, that Agesilaus is really a deep-dyed rogue---"

"Ah, never mind about Agesilaus! Perhaps he is getting them back, from
your widow at a quarter their value. Then wasn't I right when I said
that you were in possession of your bed only part of the time?"

"Yes, you were right."

"Well, and I, too, was in possession of the bed in which I died part
of the time. Proteus, the good guard of the prison, lent it to me for
a period."

"Oh, if I had known what you were aiming at with your talk, I wouldn't
have answered your wily questions. By Hercules, such profanation is
unheard of--he compares himself with me! Why, I could put an end to
you with two words, if it came to it---"

"Say them, Elpidias, without fear. Words can scarcely be more
destructive to me than the hemlock."

"Well, then, that is just what I wanted to say. You unfortunate man,
you died by the sentence of the court and had to drink hemlock!"

"But I have known that since the day of my death, even long before.
And you, unfortunate Elpidias, tell me what caused your death?"

"Oh, with me, it was different, entirely different! You see I got the
dropsy in my abdomen. An expensive physician from Corinth was called
who promised to cure me for two minas, and he was given half that
amount in advance. I am afraid that Larissa in her lack of experience
in such things gave him the other half, too---"

"Then the physician did not keep his promise?"

"That's it."

"And you died from dropsy?"

"Ah, Socrates, believe me, three times it wanted to vanquish me, and
finally it quenched the flame of my life!"

"Then tell me--did death by dropsy give you great pleasure?"

"Oh, wicked Socrates, don't make sport of me. I told you it wanted to
vanquish me three times. I bellowed like a steer under the knife of
the slaughterer, and begged the Parcæ to cut the thread of my life as
quickly as possible."

"That doesn't surprise me. But from what do you conclude that the
dropsy was pleasanter to you than the hemlock to me? The hemlock made
an end of me in a moment."

"I see, I fell into your snare again, you crafty sinner! I won't
enrage the gods still more by speaking with you, you destroyer of
sacred customs."

Both were silent, and quiet reigned. But in a short while Elpidias was
again the first to begin a conversation.

"Why are you silent, good Socrates?"

"My friend; didn't you yourself ask for silence?"

"I am not proud, and I can treat men who are worse than I am
considerately. Don't let us quarrel."

"I did not quarrel with you, friend Elpidias, and did not wish to say
anything to insult you. I am merely accustomed to get at the truth of
things by comparisons. My situation is not clear to me. You consider
your situation better, and I should be glad to learn why. On the other
hand, it would not hurt you to learn the truth, whatever shape it may
take."

"Well, no more of this."

"Tell me, are you afraid? I don't think that the feeling I now have
can be called fear."

"I am afraid, although I have less cause than you to be at odds with
the gods. But don't you think that the gods, in abandoning us to
ourselves here in this chaos, have cheated us of our hopes?"

"That depends upon what sort of hopes they were. What did you expect
from the gods, Elpidias?"

"Well, well, what did I expect from the gods! What curious questions
you ask, Socrates! If a man throughout life brings offerings, and at
his death passes away with a pious heart and with all that custom
demands, the gods might at least send some one to meet him, at least
one of the inferior gods, to show a man the way. ... But that reminds
me. Many a time when I begged for good luck in traffic in hides, I
promised Hermes calves---"

"And you didn't have luck?"

"Oh, yes, I had luck, good Socrates, but---".

"I understand, you had no calf."

"Bah! Socrates, a rich tanner and not have calves?"

"Now I understand. You had luck, had calves, but you kept them for
yourself, and Hermes received nothing."

"You're a clever man. I've often said so. I kept only three of my ten
oaths, and I didn't deal differently with the other gods. If the same
is the case with you, isn't that the reason, possibly, why we are now
abandoned by the gods? To be sure, I ordered Larissa to sacrifice a
whole hecatomb after my death."

"But that is Larissa's affair, whereas it was you, friend Elpidias,
who made the promises."

"That's true, that's true. But you, good Socrates, could you, godless
as you are, deal better with the gods than I who was a god-fearing
tanner?"

"My friend, I know not whether I dealt better or worse. At first I
brought offerings without having made vows. Later I offered neither
calves nor vows."

"What, not a single calf, you unfortunate man?"

"Yes, friend, if Hermes had had to live by my gifts, I am afraid he
would have grown very thin."

"I understand. You did not traffic in cattle, so you offered articles
of some other trade--probably a mina or so of what the pupils paid
you."

"You know, my friend, I didn't ask pay of my pupils, and my trade
scarcely sufficed to support me. If the gods reckoned on the sorry
remnants of my meals they miscalculated."

"Oh, blasphemer, in comparison with you I can be proud of my piety. Ye
gods, look upon this man! I did deceive you at times, but now and then
I shared with you the surplus of some fortunate deal. He who gives at
all gives much in comparison with a blasphemer who gives nothing.
Socrates, I think you had better go on alone! I fear that your
company, godless one, damages me in the eyes of the gods."

"As you will, good Elpidias. I swear by the dog no one shall force his
company on another. Unhand the fold of my mantle, and farewell. I will
go on alone."

And Socrates walked forward with a sure tread, feeling the ground,
however, at every step.

But Elpidias behind him instantly cried out:

"Wait, wait, my good fellow-citizen, do not leave an Athenian alone in
this horrible place! I was only making fun. Take what I said as a
joke, and don't go so quickly. I marvel how you can see a thing in
this hellish darkness."

"Friend, I have accustomed my eyes to it."

"That's good. Still I, can't approve of your not having brought
sacrifices to the gods. No, I can't, poor Socrates, I can't. The
honourable Sophroniscus certainly taught you better in your youth, and
you yourself used to take part in the prayers. I saw you."

"Yes. But I am accustomed to examine all our motives and to accept
only those that after investigation prove to be reasonable. And so a
day came on which I said to myself: 'Socrates, here you are praying to
the Olympians. Why are you praying to them?'"

Elpidias laughed.

"Really you philosophers sometimes don't know how to answer the
simplest questions. I'm a plain tanner who never in my life studied
sophistry, yet I know why I must honour the Olympians."

"Tell me quickly, so that I, too, may know why."

"Why? Ha! Ha! It's too simple, you wise Socrates."

"So much the better if it's simple. But don't keep your wisdom from
me. Tell me--why must one honour the gods?"

"Why. Because everybody does it."

"Friend, you know very well that not every one honours the gods.
Wouldn't it be more correct to say 'many'?"

"Very well, many."

"But tell me, don't more men deal wickedly than righteously?"

"I think so. You find more wicked people than good people."

"Therefore, if you follow the majority, you ought to deal wickedly and
not righteously?"

"What are you saying?"

"\emph{I'm} not saying it, \emph{you} are. But I think the reason that men
reverence the Olympians is not because the majority worship them. We
must find another, more rational ground. Perhaps you mean they deserve
reverence?"

"Yes, very right."

"Good. But then arises a new question: Why do they deserve reverence?"

"Because of their greatness."

"Ah, that's more like it. Perhaps I will soon be agreeing with you. It
only remains for you to tell me wherein their greatness consists.
That's a difficult question, isn't it? Let us seek the answer
together. Homer says that the impetuous Ares, when stretched flat on
the ground by a stone thrown by Pallas Athene, covered with his body
the space that can be travelled in seven mornings. You see what an
enormous space."

"Is that wherein greatness consists?"

"There you have me, my friend. That raises another question. Do you
remember the athlete Theophantes? He towered over the people a whole
head's length, whereas Pericles was no larger than you. But whom do we
call great, Pericles or Theophantes?"

"I see that greatness does not consist in size of body. In that you're
right. I am glad we agree. Perhaps greatness consists in virtue?"

"Certainly."

"I think so, too."

"Well, then, who must bow to whom? The small before the large, or
those who are great in virtues before the wicked?"

"The answer is clear."

"I think so, too. Now we will look further into this matter. Tell me
truly, did you ever kill other people's children with arrows?"

"It goes without saying, never! Do you think so ill of me?"

"Nor have you, I trust, ever seduced the wives of other men?"

"I was an upright tanner and a good husband. Don't forget that,
Socrates, I beg of you!"

"You never became a brute, nor by your lustfulness gave your faithful
Larissa occasion to revenge herself on women whom you had ruined and
on their innocent children?"

"You anger me, really, Socrates."

"But perhaps you snatched your inheritance from your father and threw
him into prison?"

"Never! Why these insulting questions?"

"Wait, my friend. Perhaps we will both reach a conclusion. Tell me,
would you have considered a man great who had done all these things of
which I have spoken?"

"No, no, no! I should have called such a man a scoundrel, and lodged
public complaint against him with the judges in the market-place."

"Well, Elpidias, why did you not complain in the market-place against
Zeus and the Olympians? The son of Cronos carried on war with his own
father, and was seized with brutal lust for the daughters of men,
while Hera took vengeance upon innocent virgins. Did not both of them
convert the unhappy daughter of Inachos into a common cow? Did not
Apollo kill all the children of Niobe with his arrows? Did not
Callenius steal bulls? Well, then, Elpidias, if it is true that he who
has less virtue must do honour to him who has more, then you should
not build altars to the Olympians, but they to you."

"Blaspheme not, impious Socrates! Keep quiet! How dare you judge the
acts of the gods?"

"Friend, a higher power has judged them. Let us investigate the
question. What is the mark of divinity? I think you said, Greatness,
which consists in virtue. Now is not this greatness the one divine
spark in man? But if we test the greatness of the gods by our small
human virtues, and it turns out that that which measures is greater
than that which is measured, then it follows that the divine principle
itself condemns the Olympians. But, then--"

"What, then?"

"Then, friend Elpidias, they are no gods, but deceptive phantoms,
creations of a dream. Is it not so?"

"Ah, that's whither your talk leads, you bare-footed philosopher! Now
I see what they said of you is true. You are like that fish that takes
men captive with its look. So you took me captive in order to confound
my believing soul and awaken doubt in it. It was already beginning to
waver in its reverence for Zeus. Speak alone. I won't answer any
more."

"Be not wrathful, Elpidias! I don't wish to inflict any evil upon you.
But if you are tired of following my arguments to their logical
conclusions, permit me to relate to you an allegory of a Milesian
youth. Allegories rest the mind, and the relaxation is not
unprofitable."

"Speak, if your story is not too long and its purpose is good."

"Its purpose is truth, friend Elpidias, and I will be brief. Once, you
know, in ancient times, Miletus was exposed to the attacks of the
barbarians. Among the youth who were seized was a son of the wisest
and best of all the citizens in the land. His precious child was
overtaken by a severe illness and became unconscious. He was abandoned
and allowed to lie like worthless booty. In the dead of night he came
to his senses. High above him glimmered the stars. Round about
stretched the desert; and in the distance he heard the howl of beasts
of prey. He was alone.

"He was entirely alone, and, besides that, the gods had taken from him
the recollection of his former life. In vain he racked his brain--it
was as dark and empty as the inhospitable desert in which he found
himself. But somewhere, far away, behind the misty and obscure figures
conjured up by his reason, loomed the thought of his lost home, and a
vague realisation of the figure of the best of all men; and in his
heart resounded the word 'father.' Doesn't it seem to you that the
fate of this youth resembles the fate of all humanity?"

"How so?"

"Do we not all awake to life on earth with a hazy recollection of
another home? And does not the figure of the great unknown hover
before our souls?"

"Continue, Socrates, I am listening."

"The youth revived, arose, and walked cautiously, seeking to avoid all
dangers. When after long wanderings his strength was nearly gone, he
discerned a fire in the misty distance which illumined the darkness
and banished the cold. A faint hope crept into his weary soul, and the
recollections of his father's house again awoke within him. The youth
walked toward the light, and cried: 'It is you, my father, it is you!'

"And was it his father's house?"

"No, it was merely a night lodging of wild nomads. So for many years
he led the miserable life of a captive slave, and only in his dreams
saw the distant home and rested on his father's bosom. Sometimes with
weak hand he endeavoured to lure from dead clay or wood or stone the
face and form that ever hovered before him. There even came moments
when he grew weary and embraced his own handiwork and prayed to it and
wet it with his tears. But the stone remained cold stone. And as he
waxed in years the youth destroyed his creations, which already seemed
to him a vile defamation of his ever-present dreams. At last fate
brought him to a good barbarian, who asked him for the cause of his
constant mourning. When the youth, confided to him the hopes and
longings of his soul, the barbarian, a wise man, said:

"'The world would be better did such a man and such a country exist as
that of which you speak. But by what mark would you recognise your
father?'

"'In my country,' answered the youth, 'they reverenced wisdom and
virtue and looked up to my father as to the master.'

"'Well and good,' answered the barbarian. 'I must assume that a kernel
of your father's teaching resides in you. Therefore take up the
wanderer's staff, and proceed on your way. Seek perfect wisdom and
truth, and when you have found them, cast aside your staff--there will
be your home and your father.'

"And the youth went on his way at break of day--"

"Did he find the one whom he sought?"

"He is still seeking. Many countries, cities and men has he seen. He
has come to know all the ways by land; he has traversed the stormy
seas; he has searched the courses of the stars in heaven by which a
pilgrim can direct his course in the limitless deserts. And each time
that on his wearisome way an inviting fire lighted up the darkness
before his eyes, his heart beat faster and hope crept into his soul.
'That is my father's hospitable house,' he thought.

"And when a hospitable host would greet the tired traveller and offer
him the peace and blessing of his hearth, the youth would fall at his
feet and say with emotion: 'I thank you, my father! Do you not
recognise your son?'

"And many were prepared to take him as their son, for at that time
children were frequently kidnapped. But after the first glow of
enthusiasm, the youth would detect traces of imperfection, sometimes
even of wickedness. Then he would begin to investigate and to test his
host with questions concerning justice and injustice. And soon he
would be driven forth again upon the cold wearisome way. More than
once he said to himself: 'I will remain at this last hearth, I will
preserve my last belief. It shall be the home of my father.'"

"Do you know, Socrates, perhaps that would have been the most sensible
thing to do."

"So he thought sometimes. But the habit of investigating, the confused
dream of a father, gave him no peace. Again and again he shook the
dust from his feet; again and again he grasped his staff. Not a few
stormy nights found him shelterless. Doesn't it seem to you that the
fate of this youth resembles the fate of mankind?"

"Why?"

"Does not the race of man make trial of its childish belief and doubt
it while seeking the unknown? Doesn't it fashion the form of its
father in wood, stone, custom, and tradition? And then man finds the
form imperfect, destroys it, and again goes on his wanderings in the
desert of doubt. Always for the purpose of seeking something better--"

"Oh, you cunning sage, now I understand the purpose of your allegory!
And I will tell you to your face that if only a ray of light were to
penetrate this gloom, I would not put the Lord on trial with
unnecessary questions--"

"Friend, the light is already shining," answered Socrates.



\section{V}


\lettrine[lines=3,lhang=0.11,lraise=0,loversize=0.05]{I}{}%
t seemed as if the words of the philosopher had taken effect. High up
in the distance a beam of light penetrated a vapoury envelop and
disappeared in the mountains. It was followed by a second and a third.
There beyond the darkness luminous genii seemed to be hovering, and a
great mystery seemed about to be revealed, as if the breath of life
were blowing, as if some great ceremony were in process. But it was
still very remote. The shades descended thicker and thicker; foggy
clouds rolled into masses, separated, and chased one another
endlessly, ceaselessly.

A blue light from a distant peak fell upon a deep ravine; the clouds
rose and covered the heavens to the zenith.

The rays disappeared and withdrew to a greater and greater distance,
as if fleeing from this vale of shades and horrors. Socrates stood and
looked after them sadly. Elpidias peered up at the peak full of dread.

"Look, Socrates! What do you see there on the mountain?"

"Friend," answered; the philosopher, "let us investigate our
situation. Since we are in motion, we must arrive somewhere, and since
earthly existence must have a limit, I believe that this limit is to
be found at the parting of two beginnings. In the struggle of light
with darkness we attain the crown of our endeavours. Since the ability
to think has not been taken from us, I believe that it is the will of
the divine being who called our power of thinking into existence that
we should investigate the goal of our endeavours ourselves. Therefore,
Elpidias, let us in dignified manner go to meet the dawn that lies
beyond those clouds."

"Oh, my friend! If that is the dawn, I would rather the long cheerless
night had endured forever, for it was quiet and peaceful. Don't you
think our time passed tolerably well in instructive converse? And now
my soul trembles before the tempest drawing nigh. Say what you will,
but there before us are no ordinary shades of the dead night."

Zeus hurled a bolt into the bottomless gulf.

Ctesippus looked up to the peak, and his soul was frozen with horror.
Huge sombre figures of the Olympian gods crowded on the mountain in a
circle. A last ray shot through the region of clouds and mists, and
died away like a faint memory. A storm was approaching now, and the
powers of night were once more in the ascendant. Dark figures covered
the heavens. In the centre Ctesippus could discern the all-powerful
son of Cronos surrounded by a halo. The sombre figures of the older
gods encircled him in wrathful excitement. Like flocks of birds
winging their way in the twilight, like eddies of dust driven by a
hurricane, like autumn leaves lashed by Boreas, numerous minor gods
hovered in long clouds and occupied the spaces.

When the clouds gradually lifted from the peak and sent down dismal
horror to embrace the earth, Ctesippus fell upon his knees. Later, he
admitted that in this dreadful moment he forgot all his master's
deductions and conclusions. His courage failed him; and terror took
possession of his soul.

He merely listened.

Two voices resounded there where before had been silence, the one the
mighty and threatening voice of the Godhead, the other the weak voice
of a mortal which the wind carried from the mountain slope to the spot
where Ctesippus had left Socrates.

"Are you," thus spake the voice from the clouds, "are you the
blasphemous Socrates who strives with the gods of heaven and earth?
Once there were none so joyous, so immortal, as we. Now, for long we
have passed our days in darkness because of the unbelief and doubt
that have come upon earth. Never has the mist closed in on us so
heavily as since the time your voice resounded in Athens, the city we
once so dearly loved. Why did you not follow the commands of your
father, Sophroniscus? The good man permitted himself a few little
sins, especially in his youth, yet by way of recompense, we frequently
enjoyed the smell of his offerings--"

"Stay, son of Cronos, and solve my doubts! Do I understand that you
prefer cowardly hypocrisy to searchings for the truth?"

At this question the crags trembled with the shock of a thundering
peal. The first breath of the tempest scattered in the distant gorges.
But the mountains still trembled, for he who was enthroned upon them
still trembled. And in the anxious quiet of the night only distant
sighs could be heard.

In the very bowels of the earth the chained Titans seemed to be
groaning under the blow of the son of Cronos.

"Where are you now, you impious questioner?" suddenly came the mocking
voice of the Olympian.

"I am here, son of Cronos, on the same spot. Nothing but your answer
can move me from it. I am waiting."

Thunder bellowed in the clouds like a wild animal amazed at the daring
of a Lybian tamer's fearless approach. At the end of a few moments the
Voice again rolled over the spaces:

"Son of Sophroniscus! Is it not enough that you bred so much
scepticism on earth that the clouds of your doubt reached even to
Olympus? Indeed, many a time when you were carrying on your discourse
in the market-places or in the academies or on the promenades, it
seemed to me as if you had already destroyed all the altars on earth,
and the dust were rising from them up to us here on the mountain. Even
that is not enough! Here before my very face you will not recognise
the power of the immortals--"

"Zeus, thou art wrathful. Tell me, who gave me the 'Daemon' which
spoke to my soul throughout my life and forced me to seek the truth
without resting?"

Mysterious silence reigned in the clouds.

"Was it not you? You are silent? Then I will investigate the matter.
Either this divine beginning emanates from you or from some one else.
If from you, I bring it to you as an offering. I offer you the ripe
fruit of my life, the flame of the spark of your own kindling! See,
son of Cronos, I preserved my gift; in my deepest heart grew the seed
that you sowed. It is the very fire of my soul. It burned in those
crises when with my own hand I tore the thread of life. Why will you
not accept it? Would you have me regard you as a poor master whose age
prevents him from seeing that his own pupil obediently follows out his
commands? Who are you that would command me to stifle the flame that
has illuminated my whole life, ever since it was penetrated by the
first ray of sacred thought? The sun says not to the stars: 'Be
extinguished that I may rise.' The sun rises and the weak glimmer of
the stars is quenched by its far, far stronger light. The day says not
to the torch: 'Be extinguished; you interfere with me.' The day
breaks, and the torch smokes, but no longer shines. The divinity that
I am questing is not you who are afraid of doubt. That divinity is
like the day, like the sun, and shines without extinguishing other
lights. The god I seek is the god who would say to me: 'Wanderer, give
me your torch, you no longer need it, for I am the source of all
light. Searcher for truth, set upon my altar the little gift of your
doubt, because in me is its solution.' If you are that god, harken to
my questions. No one kills his own child, and my doubts are a branch
of the eternal spirit whose name is truth."

Round about, the fires of heaven tore the dark clouds, and out of the
howling storm again resounded the powerful voice:

"Whither did your doubts tend, you arrogant sage, who renounce
humility, the most beautiful adornment of earthly virtues? You
abandoned the friendly shelter of credulous simplicity to wander in
the desert of doubt. You have seen this dead space from which the
living gods have departed. Will you traverse it, you insignificant
worm, who crawl in the dust of your pitiful profanation of the gods?
Will you vivify the world? Will you conceive the unknown divinity to
whom you do not dare to pray? You miserable digger of dung, soiled by
the smut of ruined altars, are you perchance the architect who shall
build the new temple? Upon what do you base your hopes, you who
disavow the old gods and have no new gods to take their place? The
eternal night of doubts unsolved, the dead desert, deprived of the
living spirit--\emph{this} is your world, you pitiful worm, who gnawed at
the living belief which was a refuge for simple hearts, who converted
the world into a dead chaos. Now, then, where are you, you
insignificant, blasphemous sage?"

Nothing was heard but the mighty storm roaring through the spaces.
Then the thunder died away, the wind folded its pinions, and torrents
of rain streamed through the darkness, like incessant floods of tears
which threatened to devour the earth and drown it in a deluge of
unquenchable grief.

It seemed to Ctesippus that the master was overcome, and that the
fearless, restless, questioning voice had been silenced forever. But a
few moments later it issued again from the same spot.

"Your words, son of Cronos, hit the mark better than your
thunderbolts. The thoughts you have cast into my terrified soul have
haunted me often, and it has sometimes seemed as if my heart would
break under the burden of their unendurable anguish. Yes, I abandoned
the friendly shelter of credulous simplicity. Yes, I have seen the
spaces from which the living gods have departed enveloped in the night
of eternal doubt. But I walked without fear, for my 'Daemon' lighted
the way, the divine beginning of all life. Let us investigate the
question. Are not offerings of incense burnt on your altars in the
name of Him who gives life? You are stealing what belongs to another!
Not you, but that other, is served by credulous simplicity. Yes, you
are right, I am no architect. I am not the builder of a new temple.
Not to me was it given to raise from the earth to the heavens the
glorious structure of the coming faith. I am one who digs dung, soiled
by the smut of destruction. But my conscience tells me, son of Cronos,
that the work of one who digs dung is also necessary for the future
temple. When the time comes for the proud and stately edifice to stand
on the purified place, and for the living divinity of the new belief
to erect his throne upon it, I, the modest digger of dung, will go to
him and say: 'Here am I who restlessly crawled in the dust of
disavowal. When surrounded by fog and soot, I had no time to raise my
eyes from the ground; my head had only a vague conception of the
future building. Will you reject me, you just one, Just, and True, and
Great?'"

Silence and astonishment reigned in the spaces. Then Socrates raised
his voice, and continued:

"The sunbeam falls upon the filthy puddle, and light vapour, leaving
heavy mud behind, rises to the sun, melts, and dissolves in the ether.
With your sunbeam you touched my dust-laden soul and it aspired to
you, Unknown One, whose name is mystery! I sought for you, because you
are Truth; I strove to attain to you, because you are Justice; I loved
you, because you are Love; I died for you, because you are the Source
of Life. Will you reject me, O Unknown? My torturing doubts, my
passionate search for truth, my difficult life, my voluntary
death--accept them as a bloodless offering, as a prayer, as a sigh!
Absorb them as the immeasurable ether absorbs the evaporating mists!
Take them, you whose name I do not know, let not the ghosts of the
night I have traversed bar the way to you, to eternal light! Give way,
you shades who dim the light of the dawn! I tell you, gods of my
people, you are unjust, and where there is no justice there can be no
truth, but only phantoms, creations of a dream. To this conclusion
have I come, I, Socrates, who sought to fathom all things. Rise, dead
mists, I go my way to Him whom I have sought all my life long!"

The thunder burst again--a short, abrupt peal, as if the egis had
fallen from the weakened hand of the thunderer. Storm-voices trembled
from the mountains, sounding dully in the gorges, and died away in the
clefts. In their place resounded other, marvellous tones.

When Ctesippus looked up in astonishment, a spectacle presented itself
such as no mortal eyes had ever seen.

The night vanished. The clouds lifted, and godly figures floated in
the azure like golden ornaments on the hem of a festive robe. Heroic
forms glimmered over the remote crags and ravines, and Elpidias, whose
little figure was seen standing at the edge of a cleft in the rocks,
stretched his hands toward them, as if beseeching the vanishing gods
for a solution of his fate.

A mountain-peak now stood out clearly above the mysterious mist,
gleaming like a torch over dark blue valleys. The son of Cronos, the
thunderer, was no longer enthroned upon it, and the other Olympians
too were gone.

Socrates stood alone in the light of the sun under the high heavens.

Ctesippus was distinctly conscious of the pulse-beat of a mysterious
life quivering throughout nature, stirring even the tiniest blade of
grass.

A breath seemed to be stirring the balmy air, a voice to be sounding
in wonderful harmony, an invisible tread to be heard--the tread of the
radiant Dawn!

And on the illumined peak a man still stood, stretching out his arms
in mute ecstasy, moved by a mighty impulse.

A moment, and all disappeared, and the light of an ordinary day shone
upon the awakened soul of Ctesippus. It was like dismal twilight after
the revelation of nature that had blown upon him the breath of an
unknown life.

       *        *        *        *        *

In deep silence the pupils of the philosopher listened to the
marvellous recital of Ctesippus. Plato broke the silence.

"Let us investigate the dream and its significance," he said.

"Let us investigate it," responded the others.
